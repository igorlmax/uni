\documentclass[a4paper, margins=3cm]{homework}

\name{Tobias Eidelpes}
\course{Formale Modellierung}
\term{2015WS}
\hwnum{1}
\hwtype{Übungsblatt}
\problemtitle{Aufgabe}
\solutiontitle{Lösung}

\begin{document}

\subsection*{Lösung Aufgabe 1} 
	(a) \\
	\begin{center}
		Alle x sind y. \\
		Manche z sind x. \\
		Manche z sind y. \\
		Gültig!
	\end{center}
	Alle rationalen Zahlen sind als Bruch darstellbar. \\
	Manche reelle Zahlen sind rationale Zahlen. \\
	Manche reelle Zahlen sind als Bruch darstellbar. \\
	(b) \\
	\begin{center}
		Kein x ist ein y. \\
		Alle z sind x. \\
		Kein z ist ein y. \\
		Gültig!
	\end{center}
	Kein Parallelogram ist ein Kreis. \\
	Alle Rauten sind Parallelogramme. \\
	Keine Raute ist ein Kreis. \\
	(c) \\
	\begin{center}
		Alle x machen y. \\
		Kein z ist ein x. \\
		Kein z macht y. \\
		Gültig!
	\end{center}
	Alle Katzen haben ein Fell. \\
	Kein Mensch ist eine Katze. \\
	Kein Mensch hat ein Fell.


\subsection*{Lösung Aufgabe 2}
	(a)
	\begin{center}
		Die Erdbeeren sind süß.\\
		x = Die Erdbeeren\\
		y = sind süß \\
		$x = y$
	\end{center}
	\newpage
	(b)
	\begin{center}
		Ich gehe ins Kino oder ich bleibe daheim. \\
		x = Gehe ins Kino \\
		y = bleibe daheim \\
		$x \vee y$
	\end{center}
	(c)
	\begin{center}
		Wenn ich mich nicht beeile, werde ich die Vorlesung versäumen. \\
		x = Ich beeile mich \\
		y = ich versäume die Vorlesung \\
		$\neg x \supset y$ 
	\end{center}
	(d)
	\begin{center}
		Nur wenn ich jetzt losfahre komme ich rechtzeitig zum Flughafen. \\
		x = jetzt losfahre \\
		y = komme rechtzeitig \\
		$x \supset y$
	\end{center}
	(e)
	\begin{center}
		Entweder fahre ich im Juli oder im August auf Urlaub. Beides geht sich zeitlich nicht
		aus. \\
		x = fahre im Juli \\
		y = fahre im August \\
		$x \not\equiv y$
	\end{center}
	(f)
	\begin{center}
		Ich koche heute nicht, lasse mir jedoch eine Pizza liefern. \\
		x = Ich koche heute \\
		y = lass Pizza liefern \\
		$\neg x \supset y$
	\end{center}
	(g)
	\begin{center}
		Wenn der Bus nicht rechtzeitig kommt, so werde ich nicht pünktlich sein. \\
		x = Bus kommt rechtzeitig \\
		y = ich bin pünktlich \\
		$\neg x \supset \neg y$
	\end{center}
	(h)
	\begin{center}
		Ich putze nur dann die Fenster, wenn es nicht regnet. \\
		x = Putze die Fenster \\
		y = es regnet \\
		$x \supset \neg y$
	\end{center}

\newpage

\subsection*{Lösung Aufgabe 5}
(a)
$A, B \text{ und } C$ sind Formeln. \\
$A \wedge B$ ist eine Formel. \\
Wenn $A\wedge B$ und $C$ Formeln sind, so ist auch $(A\wedge B) \supset C$ eine Formel.\\
$B \supset C$ ist eine Formel. \\
$A \supset (B \supset C)$ ebenfalls. \\
$(((A \wedge B) \supset C) \equiv (A \supset (B \supset C)))$ ist eine Formel. \\

(b) \\
$I(A) = 0,\quad I(B)=1,\quad I(C)=1$ \\
$A \wedge B\quad 0$ \\
$0 \supset C \quad 1$ \\
$B \supset C \quad 1$ \\
$A \supset 1 \quad 1$ \\
$1 \equiv 1 \quad 1$ \\
$val_I(F)=1$ \\

(c)
\begin{center}
	\begin{tabular}{ccc|ccccc|c|ccccc}
	A & B & C & ((A & $\wedge$ & B) & $\supset$ & C) & $\equiv$ & (A & $\supset$ & (B & $\supset$ & C)) \\ \hline
	1 & 1 & 1 & 1   & 1        & 1  & 1         & 1  & 1        & 1  & 1         & 1  & 1         & 1   \\
	1 & 1 & 0 & 1   & 1        & 1  & 0         & 0  & 1        & 1  & 0         & 1  & 0         & 0   \\
	1 & 0 & 1 & 1   & 0        & 0  & 1         & 1  & 1        & 1  & 1         & 0  & 1         & 1   \\
	1 & 0 & 0 & 1   & 0        & 0  & 1         & 0  & 1        & 1  & 1         & 0  & 1         & 0   \\
	0 & 1 & 1 & 0   & 0        & 1  & 1         & 1  & 1        & 0  & 1         & 1  & 1         & 1   \\
	0 & 1 & 0 & 0   & 0        & 1  & 1         & 0  & 1        & 0  & 1         & 1  & 0         & 0   \\
	0 & 0 & 1 & 0   & 1        & 0  & 1         & 1  & 1        & 0  & 1         & 0  & 1         & 1   \\
	0 & 0 & 0 & 0   & 1        & 0  & 0         & 0  & 1        & 0  & 1         & 0  & 1         & 0  
	\end{tabular}
	$val_I(F)=1 \text{ für alle } I$
	$\Rightarrow$ $F$ ist gültig und erfüllbar (Tautologie)
\end{center}

\subsection*{Lösung Aufgabe 8}
(a)
\begin{center}
	DNF: $A\wedge B \wedge \neg C$ \\
	KNF: $(\neg A \vee \neg B \vee \neg C) \wedge (A\vee \neg B \vee C)\wedge (A\vee \neg B\vee \neg C)$\\

	DNF: $(\neg A \wedge B \wedge C) \vee (\neg A \wedge B \wedge \neg C) \vee (\neg A \wedge \neg B \wedge \neg C)$ \\
	KNF: $\neg A \vee \neg B \vee C$
\end{center}

\subsection*{Lösung Aufgabe 10}
(a)
\begin{center}
	$F$ \\ 
	$A \not\equiv B$ \\
	$(B \vee E) \supset \neg W$ \\
	$A \vee B$ \\
	$A \supset (B \wedge F)$
\end{center}
(b)
Die Zutaten sind $B$, $E$ und $F$. $F$, damit der Trank dickflüssig wird, nach Aussage
Nummer 4 kommt $E$ dazu, $A$ kommt jedoch durch Aussage 1 nicht hinein, dafür aber $B$
nach Aussage 2.
\end{document}