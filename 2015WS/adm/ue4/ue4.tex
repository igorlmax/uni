\documentclass[a4paper, 12pt, margins=3cm]{homework}
\usepackage{tikz}

\usepackage{graphicx}
\usepackage{dsfont}
\usepackage{microtype}
\usepackage{mathrsfs}
\usepackage[ngerman]{babel}
\usepackage{csquotes}
\usepackage[T1]{fontenc}
\usepackage{lmodern}
\usepackage{wasysym}

\setlength{\parindent}{0pt}

\newcommand{\R}{\mathbb{R}}
\newcommand{\N}{\mathbb{N}}
\newcommand{\Z}{\mathbb{Z}}
\newcommand{\Q}{\mathbb{Q}}

\name{Tobias Eidelpes}
\course{Algebra und Diskrete Mathematik}
\term{2015WS}
\hwnum{4}
\hwtype{Übungsblatt}
\problemtitle{Aufgabe}
\solutiontitle{Lösung}

\begin{document}
  \begin{center}
    \textsc{Beispiele 138, 142, 155}
  \end{center}

  \problemnumber{138}
  \begin{problem}
    Seien $f:A\rightarrow B$ und $g:B\rightarrow C$ Abbildungen. Zeigen Sie, dass
    aus der Surjektivität von $g\circ f$ die Surjektivität von $g$ und aus der
    Injektivität von $g\circ f$ die Injektivität von $f$ folgt.
  \end{problem}
  \begin{solution}
    Aus der Surjektivität von $g\circ f$ folgt, dass es für jedes $c\in C$ ein 
    $a\in A$ geben muss, für das gilt: $(g\circ f)(a) = g(f(a)) = c$. $f(a)$ wird
    mit b substituiert und es gilt: $g(b) = c$. Damit $g\circ f$ surjektiv sein
    kann, muss $g$ surjektiv sein, damit jedem $c\in C$ ein $b\in B$ zugeordnet
    wird. $f$ muss aber auch surjektiv sein, da sonst $b$ nicht jedes Element
    aus $B$ sein kann. \\

    Aus der Injektivität von $g\circ f$ folgt
    \[ (g\circ f)(a) = g(f(a)) = g(f(b)) = (g\circ f)(b)\Rightarrow a = b \]
    Für $f(a)$ setzen wir $m$ ein und für $f(b)$ $n$. Daraus folgt:
    \[ (a = b) \Longleftrightarrow f(a) = f(b) \Longleftrightarrow m = n \]
    In die obige Formel für $f(a)$ $m$, für $f(b)$ $n$ und für $(a = b)$ $(m = n)$
    eingesetzt:
    \[ g(m) = g(n) \Rightarrow m = n \] 
    Somit ist $g$ injektiv. Dadurch gilt:
    \[ g(f(a)) = g(f(b)) \Rightarrow f(a) = f(b) \text{ außerdem}\]
    \[ g(f(a)) = g(f(b)) \Rightarrow a = b \]
    Deshalb gilt auch
    \[f(a) = f(b) \Rightarrow a = b\]
    was wiederum aussagt, dass $f$ ebenfalls injektiv ist.
  \end{solution}

  \problemnumber{142}
  \begin{problem}
    Man zeige, dass die Funktion $f: \R \backslash  \{6\} \rightarrow \R \backslash  \{-10\}$,
    $y = \frac{10x+1}{6-x}$ bijektiv ist und bestimme ihre Umkehrfunktion.
  \end{problem}
  \begin{solution}
    \begin{align*}
      y &= \frac{10x+1}{6-x} \\
      y\cdot (6-x) &= 10x+1 \\
      6y - xy &= 10x+1 \\
      6y-1 &= 10x+xy \\
      6y-1 &= x\cdot (10+y) \\
      x &= \frac{6y-1}{10+y}
    \end{align*}
    Umkehrfunktion: $f^{-1}(x)=\frac{6x-1}{10+x}$ \\

    Injektivität wird dadurch überprüft, dass zwei idente Funktionswerte den 
    gleichen Ausgangswert besitzen. \\
    $f(x_1) = f(x_2) \Longrightarrow x_1 = x_2$
    \begin{align*}
      \frac{10x_1+1}{6-x_1} &= \frac{10x_2+1}{6-x_2} \\
      (10x_1+1) \cdot (6-x_2) &= (10x_2+1) \cdot (6-x_1) \\
      60x_1 - 10x_1x_2+6-x_2 &= 60x_2-10x_1x_2+6-x_1 \\
      60x_1-x_2 &= 60x_2-x_1 \\
      61x_1 &= 61x_2 \\
      x_1 &= x_2
    \end{align*}

    Surjektivität wird dadurch überprüft, dass die Funktion für jeden Funktionswert
    mindestens einen Ausgangswert haben muss.\\

    \[ f^{-1} = \frac{6x-1}{10+x} \in \R \backslash \{-10\} \quad \text{\checkmark} \]
  \end{solution}
  
  \problemnumber{155}
  \begin{problem}
      Wieviele \glqq Wörter\grqq\ der Länge 28 gibt es, bei denen genau 5-mal
      der Buchstabe \texttt{a}, 14-mal \texttt{b}, 5-mal \texttt{c}, 3-mal \texttt{d}
      vorkommen und genau einmal \texttt{e} vorkommt?
  \end{problem}
  \begin{solution}
    \[ 
      \frac{(k_1+k_2+...+k_n)!}{k_1! \cdot k_2! \cdot ... \cdot k_n!} = 
      \frac{(5+14+5+3+1)!}{5!\cdot 14! \cdot 5! \cdot 3! \cdot 1!}
      \approx 4.048\cdot 10^{13}
    \]
  \end{solution}
\end{document}