\documentclass[a4paper, margins=2.5cm]{homework}

\usepackage{dsfont}
\usepackage{microtype}
\usepackage{mathrsfs}
\usepackage[ngerman]{babel}
\usepackage{csquotes}
\usepackage[T1]{fontenc}
\usepackage{lmodern}

\newcommand{\R}{\mathbb{R}}
\newcommand{\N}{\mathbb{N}}
\newcommand{\Z}{\mathbb{Z}}
\newcommand{\Q}{\mathbb{Q}}

\name{Tobias Eidelpes}
\course{Algebra und Diskrete Mathematik}
\term{2015WS}
\hwnum{2}
\hwtype{Übungsblatt}
\problemtitle{Aufgabe}
\solutiontitle{Lösung}

\begin{document}
	\begin{center}
    \textsc{Beispiele 11, 12, 43, 47, 53, 59, 64}
  \end{center}

\problemnumber{11}
\begin{problem}
	Handelt es sich bei der aussagenlogischen Formel
	\[ [(A \rightarrow B) \wedge (B \rightarrow C)] \leftrightarrow (A \rightarrow C) \]
	um eine Tautologie, um eine Kontradiktion oder um eine erfüllbare Formel?
\end{problem}
\begin{solution}
	Es handelt sich um eine erfüllbare Formel, da die Aussage für mindestens
	eine Interpretation wahr ist. \\

	\begin{center}
		\begin{tabular}{|ccc|c|c|c|c|c|}
			\hline
			$A$ & $B$ & $C$ & $[(A \rightarrow B)$ & $\wedge$ & $(B \rightarrow C]$ & $\leftrightarrow$ & $(A \rightarrow C)$ \\ \hline
			1   & 1   & 1   & 1                    & 1        & 1                   & \textbf{1}        & 1                   \\ \hline
			1   & 1   & 0   & 1                    & 0        & 0                   & \textbf{1}        & 0                   \\ \hline
			1   & 0   & 1   & 0                    & 0        & 1                   & \textbf{0}        & 1                   \\ \hline
			1   & 0   & 0   & 0                    & 0        & 1                   & \textbf{1}        & 0                   \\ \hline
			0   & 1   & 1   & 1                    & 1        & 1                   & \textbf{1}        & 1                   \\ \hline
			0   & 1   & 0   & 1                    & 0        & 0                   & \textbf{0}        & 1                   \\ \hline
			0   & 0   & 1   & 1                    & 1        & 1                   & \textbf{1}        & 1                   \\ \hline
			0   & 0   & 0   & 1                    & 1        & 1                   & \textbf{1}        & 1                   \\ \hline
		\end{tabular}
	\end{center}
\end{solution}

\problemnumber{12}
\begin{problem}
	Gelten folgende Formeln? Geben Sie jeweils eine verbale Begründung.
	\begin{parts}
		\part
		\label{12.a}
		$\forall x \in \N \; \exists y \in \N : x < y$
		\part
		\label{12.b}
		$\exists y \in \N \; \forall x \in \N : x < y$
		\part
		\label{12.c}
		$\forall x \in \N \; \exists y \in \N : y < x$
		\part
		\label{12.d}
		$\forall x \in \Z \; \exists y \in \Z : y < x$
	\end{parts}
\end{problem}
\begin{solution}
	\ref{12.a}
	Aussage stimmt, weil die natürlichen Zahlen unendlich sind und somit
	jede Zahl einen Nachfolger hat. \\
	\ref{12.b}
	Aussage ist falsch, weil es einerseits keine negativen Zahlen in $\N$ gibt
	und damit beispielsweise $0$ keinen Vorgänger hat, andererseits, weil
	es nie eine Zahl geben kann, die größer als alle anderen ist, da $\N$ unendlich
	ist. \\
	\ref{12.c}
	Aussage ist falsch, da die Zahl $0$ keinen Vorgänger hat. \\
	\ref{12.d}
	Aussage ist wahr, da jede Zahl in $\Z$ einen Vorgänger hat.
\end{solution}

\newpage

\problemnumber{43}
\begin{problem}
	Man bestimme den ggT(1109,4999) mit Hilfe des Euklidischen Algorithmus.
\end{problem}
\begin{solution}
	\begin{align*}
		4999 \text{ mod } 1109 &= 563 \\
		1109 \text{ mod } 563  &= 546 \\
		563 \text{ mod } 546   &= 17 \\
		546 \text{ mod } 17    &= 2 \\
		17 \text{ mod } 2      &= 1
	\end{align*}
	\[ \text{ggT(1109, 4999)} = 1 \]
\end{solution}

\problemnumber{47}
\begin{problem}
	Man bestimme zwei ganze Zahlen $x,y$, welche die Gleichung
	$243x +198y = 9$ erfüllen.
\end{problem}
\begin{solution}
	\begin{align*}
		243 &= 1 \cdot 198 + 45 \\
		198 &= 4 \cdot 45 + 18 \\
		45 &= 2 \cdot 18 + 9
	\end{align*}
	\[ \text{ggT(243, 198)} = 9 \]
	Erweiterter Euklidischer Algorithmus:
	\begin{align*}
		9 &= 45 - 2 \cdot (198 - 4 \cdot 45) \\
		9 &= -(2 \cdot 198 - 9 \cdot 45) \\
		9 &= -2 \cdot 198 + 9 \cdot 45 \\
		9 &= -2 \cdot 198 + 9 \cdot (243-198) \\
		9 &= 9 \cdot 243 - 11 \cdot 198
	\end{align*}
	$\Longrightarrow$ $x=9$ und $y=-11$
\end{solution}

\problemnumber{53}
\begin{problem}
	Bestimmen Sie alle Lösungen der folgenden Kongruenzen bzw. beweisen Sie die
	Unlösbarkeit (in $\Z$):
	\begin{parts}
		\part
		\label{53.a}
		$6x \equiv 3 \text{ mod } 9$
		\part
		\label{53.b}
		$6x \equiv 4 \text{ mod } 9$
	\end{parts}
\end{problem}
\begin{solution}
	\ref{53.a}
	Lösbar, weil ggT(6, 9) = 3 und 3 ein Teiler von 3 ist.
	Nach erweitertem euklidischen Algorithmus gilt $3=-1 \cdot 6 + 1 \cdot 9$.
	Daraus folgt, dass $x \equiv -1 + k \cdot \frac{9}{ggT(6,9)}$, also
	$x \equiv -1 + k \cdot 3$ \\

	\ref{53.b}
	Nicht lösbar, weil ggT(6,9) = 3 und 3 kein Teiler von 4 ist. \\
	$\Longrightarrow$ Wenn $ggT(a,m)|b$, dann kann die Kongruenz durch den größten
	gemeinsamen Teiler dividiert werden und es folgt daraus eine neue Kongruenz
	mit $ggT(a',m')=1$
	$\Longrightarrow$ es gibt eine Zahl $c$, sodass $a' \cdot c = 1 \text{ mod } m'$
	ist. Da die Voraussetzung aber nicht gegeben ist, gibt es auch kein Inverses
	und daraus folgt die Unlösbarkeit.
\end{solution}

\problemnumber{59}
\begin{problem}
	Im europäischen Artikelnummernsystem EAN werden Zahlen mit 13 Dezimalziffern
	der Form $a_1\, a_2\, …\, a_{12}\, p$ verwendet. Dabei wird die letzte der
	13 Ziffern, das ist die Prüfziffer $p$ im EAN-Code so bestimmt, dass
	\[ a_1+3a_2+a_3+3a_4+…+a_{11}+3a_{12} + p \equiv 0 \text{ mod } 10 \]
	gilt. Man zeige, dass beim EAN-Code ein Fehler in einer einzelnen Ziffer stets
	erkannt wird, während eine Vertauschung von zwei benachbarten Ziffern genau dann
	nicht erkannt wird, wenn die beiden Ziffern gleich sind oder sich um 5 unterscheiden.
\end{problem}
\begin{solution}
	Wenn ein Fehler in einer einzelnen Ziffer auftritt, dann ist die Summe der
	Ziffern eine Andere und ergibt nach dem Addieren der Prüfziffer $p$ nicht mehr
	$0 \text{ mod } 10$. Wenn zwei benachbarte, gleiche Ziffern vertauscht werden,
	ändert sich nichts an der Summe und der Fehler wird somit nicht erkannt. Wenn
	sich die beiden Ziffern um 5 unterscheiden, ändert sich ebenfalls nichts an
	dem Ergebnis, weil
	\begin{align*}
		\overbrace{x+3(x+5) \text{ mod } 10}^\text{Richtig} & \equiv
		\overbrace{(x+5)+3x \text{ mod } 10}^\text{Vertauscht} \\
		4x+15 \text{ mod } 10 & \equiv 4x+5 \text{ mod } 10 \\
		4x+5 & \equiv 4x+5
	\end{align*}
	Das heißt die Vertauschung verändert die Summe in diesem Fall nicht und der Fehler
	bleibt unerkannt.
\end{solution}

\problemnumber{64}
\begin{problem}
	Man zeige, dass $n^3-n$ für alle $n \in \N$ stets durch 3 teilbar ist, mittels
	\begin{parts}
		\part
		\label{64.a}
		eines direkten Beweises,
		\part
		\label{64.b}
		eines Beweises durch vollständige Induktion.
	\end{parts}
\end{problem}
\begin{solution}
	\ref{64.a}
	\[ n^3-n = n \cdot (n^2-1) = (n-1) \cdot (n+1) \]
	$\Longrightarrow$ Das Produkt von drei aufeinanderfolgenden Zahlen, von denen
	eine zwangsläufig durch 3 teilbar ist.  \\
	$\Longrightarrow$ Das ganze Produkt ist durch 3 teilbar. \\
	$\Longrightarrow$ Somit ist bewiesen, dass $n^3-n$ für alle $n \in \N$ durch
	3 teilbar ist.

	\newpage

	\ref{64.b}
	Induktionsanfang: \[ \text{Für $n = 1$: }\qquad 1^3 - 1 = 0\qquad 3\, \vert \, 0 \]
	Induktionsschritt: \[ A(n) \rightarrow A(n+1) \]
	Induktionsvoraussetzung: \[ A(n) \text{ gilt für ein bestimmtes } n \in \N \]
	Induktionsbehauptung: \[ A(n+1) \text{ gilt} \]
	\[ (n+1)^3 - (n+1) \Longleftrightarrow n^3 + 3n^2 + 2n \Longleftrightarrow
	(n^3-n) + 3n^2 + 3n \Longleftrightarrow (n^3-n) + 3(n^2+n) \]
	$\Longrightarrow n^3-n$ ist laut Induktionsannahme durch 3 teilbar. \\
	$\Longrightarrow 3(n^2+n)$ ist aufgrund des Faktors 3 durch 3 teilbar. \\
	$\Longrightarrow (n+1)^3 - (n+1)$ ist durch 3 teilbar. \\
	$\Longrightarrow n^3-n$ ist durch 3 teilbar für alle $n \in \N$.
\end{solution}

\end{document}