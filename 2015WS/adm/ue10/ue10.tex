\documentclass[a4paper, 12pt, margins=3cm]{homework}
\usepackage{tikz}

\usepackage{graphicx}
\usepackage{dsfont}
\usepackage{microtype}
\usepackage{mathrsfs}
\usepackage[ngerman]{babel}
\usepackage{csquotes}
\usepackage[T1]{fontenc}
\usepackage{lmodern}
\usepackage{wasysym}

\setlength{\parindent}{0pt}

\newcommand{\R}{\mathbb{R}}
\newcommand{\N}{\mathbb{N}}
\newcommand{\Z}{\mathbb{Z}}
\newcommand{\Q}{\mathbb{Q}}
\newcommand{\C}{\mathbb{C}}

\name{Tobias Eidelpes}
\course{Algebra und Diskrete Mathematik}
\term{2015WS}
\hwnum{10}
\hwtype{Übungsblatt}
\problemtitle{Aufgabe}
\solutiontitle{Lösung}

\begin{document}
  \begin{center}
    \textsc{Beispiele 408, 422, 437, 442, 458, 474, 476}
  \end{center}


% AUSSTÄNDIG 408
  \problemnumber{408}
  \begin{problem}
    Untersuchen Sie, ob die folgende Struktur ein Ring, Integritätsring bzw. Körper
    ist:\\

    $M = \Q[\sqrt{7}] = \{ a+b\sqrt{7}\;|\;a,b\in \Q \}$ mit der Addition und
    Multiplikation aus $\R$. 
  \end{problem}
  \begin{solution}\hfill
    \subsection*{Addition}

      \subsubsection*{Abgeschlossenheit}
        \[ (w+x\cdot\sqrt{7}) + (y+z\cdot\sqrt{7}) = (w+y) + ((x+z)\cdot\sqrt{7}) \]
        \[ w+y\in\Q, x+z\in\Q \Longrightarrow \text{abgeschlossen} \]

      \subsubsection*{Assoziativität}
        \[ a\circ (b\circ c) = (a\circ b)\circ c \]
        \[ a = (u+v\cdot\sqrt{7}) \]
        \[ b = (w+x\cdot\sqrt{7}) \]
        \[ c = (y+z\cdot\sqrt{7}) \]
        \begin{align*}
          (u+v\cdot\sqrt{7}) + ((w+x\cdot\sqrt{7}) + (y+z\cdot\sqrt{7})) &= ((u+v\cdot\sqrt{7}) + (w+x\cdot\sqrt{7})) + (y+z\cdot\sqrt{7})\\ 
          (u+w+y) + (v+x+z)\cdot\sqrt{7} &= (u+w+y) + (v+x+z)\cdot\sqrt{7}
        \end{align*}
        \[ \Longrightarrow \text{assoziativ} \]

      \subsubsection*{Neutrales Element}
        \[ e = 0+0\cdot\sqrt{7}=0 \]

      \subsubsection*{Inverses Element}
        \[ (x+y\cdot\sqrt{7})^{-1} = (-x) + (-y)\cdot\sqrt{7} \]

      \subsubsection*{Kommutativität}
        Gegeben durch die Kommutativität der Addition und Multiplikation.
  \end{solution}


% AUSSTÄNDIG 442
  \problemnumber{422}
  \begin{problem}
    Sei $\langle R,+,\cdot\rangle$ ein Ring. Man zeige, dass dann auch $R\times R$
    mit den Operationen 
    \[ (a,b) + (c,d) = (a+c,b+d) \]
    \[ (a,b)\cdot (c,d) = (a\cdot c,b\cdot d) \]
    ein Ring ist.
  \end{problem}
  \begin{solution}
    
  \end{solution}


% AUSSTÄNDIG 437
  \problemnumber{437}
  \begin{problem}
    Man untersuche das Polynom $x^2 + x + 1$ auf Irreduzibilität a) über $\Q$,
    b) über $\Z_3$.
  \end{problem}
  \begin{solution}
    
  \end{solution}


% AUSSTÄNDIG 442
  \problemnumber{442}
  \begin{problem}
    Man zeige, dass die folgende algebraische Struktur ein Verband ist. Ist dieser
    distributiv oder eine Boolsche Algebra?
    \begin{parts}
      \part $(\mathbf{P}(A), \cap , \cup)$
      \part $(\{ X\subseteq \N \;|\; X\text{ ist endlich oder }\N\,\backslash\, X\text{ ist endlich} \}, \cap, \cup)$
    \end{parts}
  \end{problem}
  \begin{solution}
    
  \end{solution}


% AUSSTÄNDIG 458
  \problemnumber{458}
  \begin{problem}
    Untersuchen Sie, ob $W$ Teilraum des Vektorraums $\R^3$ über $\R$ ist und
    beschreiben Sie die Menge $W$ geometrisch:
    \[ W = \{ (x,y,z) \;|\; x+y+z\geq 0 \} \]
  \end{problem}
  \begin{solution}
    
  \end{solution}


% AUSSTÄNDIG 474
  \problemnumber{474}
  \begin{problem}
    Zeigen Sie: Die Menge aller Polynome $a_0 + a_1x + a_2x^2 + a_3x^3 + a_4x^4$
    vom Grad kleiner gleich 4 mit Koeffizienten $a_i$ aus $\Q$ bildet mit der
    üblichen Addition und dem üblichen Produkt mit einem Skalar einen Vektorraum
    über $\Q$.
  \end{problem}
  \begin{solution}
    
  \end{solution}


% AUSSTÄNDIG 476
  \problemnumber{476}
  \begin{problem}
    Bestimmen Sie den kleinsten Teilraum des Vektorraumes aus 474) der die Polynome
    $x-x^2$ und $x+x^3$ enthält.
  \end{problem}
  \begin{solution}
    
  \end{solution}
\end{document}