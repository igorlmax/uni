\documentclass[a4paper, 12pt, margins=2.5cm]{homework}
\usepackage{tikz}

\usepackage{graphicx}
\usepackage{dsfont}
\usepackage{microtype}
\usepackage{mathrsfs}
\usepackage[ngerman]{babel}
\usepackage{csquotes}
\usepackage[T1]{fontenc}
\usepackage{lmodern}
\usepackage{wasysym}

\setlength{\parindent}{0pt}

\newcommand{\R}{\mathbb{R}}
\newcommand{\N}{\mathbb{N}}
\newcommand{\Z}{\mathbb{Z}}
\newcommand{\Q}{\mathbb{Q}}
\newcommand{\C}{\mathbb{C}}

\name{Tobias Eidelpes}
\course{Algebra und Diskrete Mathematik}
\term{2015WS}
\hwnum{5}
\hwtype{Übungsblatt}
\problemtitle{Aufgabe}
\solutiontitle{Lösung}

\begin{document}
  \begin{center}
    \textsc{Beispiele 168, 170, 173, 178, 186, 191, 200}
  \end{center}

% ERLEDIGT
  \problemnumber{168}
  \begin{problem}
    Wie viele verschiedene Tipps müssen beim Lotto \glqq 6 aus 45\grqq\ abgegeben
    werden, um sicher einen Sechser zu erzielen? Wie viele verschiedene Tipps 
    führen zu keinem Gewinn (d.h., diese Tipps enthalten maximal zwei richtige
    Zahlen), bei wie vielen möglichen Tipps stimmt mindestens eine Zahl, bei wie
    vielen sind alle Zahlen falsch?
  \end{problem}
  \begin{solution}
    Um sicher einen Sechser zu erzielen, müssen von $\binom{45}{6}$ möglichen
    Tipps $\binom{45}{6}$ Tipps abgegeben werden, da einer von allen Tipps ein
    Sechser ist.
    \begin{itemize}\itemsep0pt
      \item Mögliche Tipps ohne richtige Zahlen: $\binom{6}{0}\cdot \binom{39}{6} = 3262623$
      \item Mögliche Tipps mit einer richtigen Zahl: $\binom{6}{1}\cdot \binom{39}{5} = 3454542$
      \item Mögliche Tipps mit zwei richtigen Zahlen: $\binom{6}{2}\cdot \binom{39}{4} = 1233765$
    \end{itemize}
    Nun müssen die einzelnen Tippmöglichkeiten addiert werden:
    \[
      \binom{6}{0}\cdot \binom{39}{6} + \binom{6}{1}\cdot \binom{39}{5} +
      \binom{6}{2}\cdot \binom{39}{4} = 7950930
    \]
    
  \end{solution}


% ERLEDIGT
  \problemnumber{170}
  \begin{problem}
    Wieviele natürliche Zahlen $n < 100\,000$ enthalten in ihrer Dezimalentwicklung
    genau dreimal die Ziffer drei?
  \end{problem}
  \begin{solution}
    Da wir alle Zahlen von $0-99\,999$ betrachten, gibt es $100\,000$ Möglichkeiten.
    $3$ von $5$ Ziffern müssen einen dreier enthalten, $2$ Ziffern dürfen alle
    Zahlen von $0-2$ und $4-9$ annehmen. \\

    Alle möglichen Kombinationen, die an $3$ von $5$ Stellen dreier enthalten
    bekommt man durch
    \[ \binom{5}{3} = \frac{5!}{3! \cdot (5-3)!} = \frac{120}{12} = 10. \]
    Nun werden die restlichen zwei Ziffern mit Kombinationen von $0-2$ und $4-9$
    aufgefüllt, was 9 Möglichkeiten pro Ziffer entspricht. Das für beide Ziffern ist
    \[ 9\cdot 9 = 81\]
    multipliziert mit $10$:
    \[ 10 \cdot 81 = 810 \text{ Kombinationen} \]
  \end{solution}


% AUSSTÄNDIG
  \problemnumber{173}
  \begin{problem}
    Man beweise die Formel
    \[ \binom{2n}{n} = \sum_{k=0}^n{\binom{n}{k}^2} = \sum_{k=0}^n{\binom{n}{k}\binom{n}{n-k}}. \]
    (Hinweis: Man betrachte die Koeffizienten von $(1+x)^n (1+x)^n = (1+x)^{2n}$.)
  \end{problem}
  \begin{solution}
    
  \end{solution}


% ERLEDIGT
  \problemnumber{178}
  \begin{problem}
    Berechnen Sie unter Benützung des Binomischen Lehrsatzes (und ohne Benützung
    der Differentialrechnung):
    \[ \sum_{k=0}^{n}{\binom{n}{k}k5^k} \]
  \end{problem}
  \begin{solution}
    Dafür muss der binomische Lehrsatz 
    \[ \sum_{k=0}^{n} \binom{n}{k} x^{n-k} y^k = (x+y)^n \]
    so umgeformt werden, dass er auf die Angabe angewendet werden kann.
    \[ \sum_{k=1}^{n}{k \binom{n}{k} 5^k} \]
    Nun wird $k$ ersteinmal in den Binomialkoeffizienten hineinmultipliziert.
    \[  k\binom{n}{k} = \frac{n!}{k!(n-k)!}\cdot k = \frac{k\cdot n!}{k!(n-k!)} = 
        \frac{k\cdot n!}{k!\cdot k\cdot (n-1-k)!} = \frac{n!}{k!(n-1-k)!} =
        n\binom{n-1}{k-1}
    \]
    Anschließend wird in die Angabe eingesetzt:
    \[
      \sum_{k=1}^{n}{n\binom{n-1}{k-1} 5^k} = n\sum_{k=1}^{n}{\binom{n-1}{k-1} 5^k} =
      \underbrace{n\sum_{k=0}^{n-1}{n\binom{n-1}{k} 5^{k+1}}}_\text{Indexverschiebung $k=k-1$} = 5n\sum_{k=0}^{n-1}{\binom{n-1}{k} 5^k}
    \]
    Jetzt in binomischen Lehrsatz einsetzen mit $x=1$, $y=5$ und $n=n-1$.
    \[
      \sum_{k=0}^{n-1}{\binom{n-1}{k} 1^{n-1-k} 5^k} = \sum_{k=0}^{n-1}{\binom{n-1}{k} 5^k} =
      (1+5)^{n-1}
    \]
    Das Ergebnis:
    \[ 5n\sum_{k=0}^{n-1}{\binom{n-1}{k} 5^k} = 5n (1+5)^{n-1} \]

  \end{solution}


% ERLEDIGT
  \problemnumber{186}
  \begin{problem}
    Zeigen Sie mithilfe des Schubfachprinzips: Unter je 15 natürlichen Zahlen
    gibt es mindestens zwei, deren Differenz durch 14 teilbar ist.
  \end{problem}
  \begin{solution}
    Man betrachte alle Restklassen Modulo $14$:
    \[ x \equiv y\; \text{ (mod) }m \Leftrightarrow m\; |\; x-y \]
    Nun gibt man jeder Restklasse ein Schubfach also bekommt man 14 Fächer. Da
    wir 15 Zahlen haben, können wir 14 im besten Fall so wählen, dass jede der 
    14 Zahlen in einem Schubfach landet. Die letzte Zahl muss nun wohl oder übel
    in ein Schubfach in dem schon eine Zahl ist. Dadurch liegen zwei Zahlen in
    einer Restklasse und daraus folgt, dass ihre Differenz durch 14 teilbar ist.
  \end{solution}


% AUSSTÄNDIG
  \problemnumber{191}
  \begin{problem}
    Wieviele natürliche Zahlen $n$ mit $1\leq n \leq 10^6$ gibt es, die weder
    Quadrat, noch dritte, vierte oder fünfte Potenz einer natürlichen Zahl sind?
  \end{problem}
  \begin{solution}
    
  \end{solution}


% ERLEDIGT
  \problemnumber{200}
  \begin{problem}
    Man bestimme die Anzahl aller Anordnungen (Permutationen) der Buchstaben 
    \texttt{a,b,c,d,e,f,g,h,} in denen weder der Block \glqq \texttt{acg}\grqq\ 
    noch der Block \glqq \texttt{cgbe}\grqq\ vorkommt. (Hinweis: Die Anzahl der
    Permutationen einer $n$-elementigen Menge ist $n!$.)
  \end{problem}
  \begin{solution}
    Alle möglichen Anordnungen ohne Berücksichtigung der Aufgabenstellung:
    \[ 8! = 40320 \]
    weil wir acht Buchstaben (\texttt{a-h}) haben. Für den Block \texttt{acg}
    ergeben sich sechs mögliche Anordnungen mit fünf anordbaren Buchstaben, das heißt:
    \[ 6\cdot 5! = 720 \]
    Für den Block \texttt{cgbe} gibt es fünf mögliche Anordnungen mit vier 
    anordbaren Buchstaben, das heißt:
    \[ 5\cdot 4! = 120 \]
    Jetzt gibt es aber auch Anordnungen, in denen beide Blöcke vorkommen, das
    heißt Anordnungen nach dem Schema \texttt{acgbe}. Hier sind vier Anordnungen
    möglich mit drei anordbaren Buchstaben:
    \[ 4\cdot 3! = 24 \]
    Aus diesen Mengen von Anordnungen muss nun die korrekte Anzahl an Anordnungen
    ermittelt werden. Dazu müssen von der Gesamtmenge ($8!$) die Anordnungen für
    Block \texttt{acg} und für Block \texttt{cgbe} subtrahiert und die 
    Möglichkeiten für Block \texttt{acgbe} addiert werden. Das muss deshalb
    geschehen, weil sonst die Anordnungen in denen beide Blöcke enthalten sind
    am Anfang nur einmal \glqq hineinkommen\grqq\ ($8!$), aber zweimal abgezogen
    werden, nämlich einmal bei Block \texttt{acg} und das zweite Mal bei Block
    \texttt{cgbe}. Also:
    \[ 8! - 6\cdot 5! - 5\cdot 4! + 4\cdot 3! = 40320 - 720 - 120 + 24 = 39504 \]
  \end{solution}
\end{document}