\documentclass[a4paper, 12pt, margins=3cm]{homework}
\usepackage{tikz}

\usepackage{graphicx}
\usepackage{dsfont}
\usepackage{microtype}
\usepackage{mathrsfs}
\usepackage[ngerman]{babel}
\usepackage{csquotes}
\usepackage[T1]{fontenc}
\usepackage{lmodern}
\usepackage{wasysym}

\setlength{\parindent}{0pt}

\newcommand{\R}{\mathbb{R}}
\newcommand{\N}{\mathbb{N}}
\newcommand{\Z}{\mathbb{Z}}
\newcommand{\Q}{\mathbb{Q}}
\newcommand{\C}{\mathbb{C}}

\name{Tobias Eidelpes}
\course{Algebra und Diskrete Mathematik}
\term{2015WS}
\hwnum{11}
\hwtype{Übungsblatt}
\problemtitle{Aufgabe}
\solutiontitle{Lösung}

\begin{document}
  \begin{center}
    \textsc{Beispiele 495, 544, 556}
  \end{center}


% AUSSTÄNDIG 495
  \problemnumber{495}
  \begin{problem}
    Untersuchen Sie, ob die folgenden Vektoren des $\Z_5^4$ linear unabhängig sind:
    $(1,2,3,4),\quad (2,3,4,1),\quad (3,4,2,1)$.
  \end{problem}
  \begin{solution}
    
  \end{solution}


% ERLEDIGT 544
  \problemnumber{544}
  \begin{problem}
    Bestimmen Sie den Rang der folgenden reellen Matrix:
    \[ 
      \begin{pmatrix}
        1 & 2 & 3 & 4 & 5 \\
        2 & 3 & 4 & 5 & 6 \\
        3 & 4 & 5 & 6 & 7 \\
        4 & 5 & 6 & 7 & 8
      \end{pmatrix}
    \]

  \end{problem}
  \begin{solution}\hfill
    \begin{enumerate}\itemsep0pt
      \item Erste Zeile mal $2$.
      \item Zweite Zeile $-$ erste Zeile.
      \item Erste Zeile dividiert durch 2.
      \item Erste Zeile mal 3.
      \item Dritte Zeile $-$ erste Zeile.
      \item Erste Zeile dividiert durch 3.
      \item Erste Zeile mal 4.
      \item Vierte Zeile $-$ erste Zeile.
      \item Zweite Zeile mal -1.
      \item Zweite Zeile mal -2.
      \item Dritte Zeile $-$ zweite Zeile.
      \item Zweite Zeile dividiert durch -2.
      \item Zweite Zeile mal -3.
      \item Vierte Zeile - zweite Zeile.
      \item Zweite Zeile mal -1.
    \end{enumerate}
    \[
      \begin{pmatrix}
        1 & 2 & 3 & 4 & 5\\
        0 & 1 & 2 & 3 & 4\\
        0 & 0 & 0 & 0 & 0\\
        0 & 0 & 0 & 0 & 0
      \end{pmatrix}
    \]
    Zwei linear unabhängige Zeilen, deshalb ist der Rang = 2.
  \end{solution}


% ERLEDIGT 556 
  \problemnumber{556}
  \begin{problem}
    Bestimmen Sie die inverse Matrix $A^{-1}$.
    \[ A =
      \begin{pmatrix}
        -1 & 3 & 2 \\
        -2 & 4 & 6 \\
        1 & -2 & 2
      \end{pmatrix}
    \]
  \end{problem}
  \begin{solution}\hfill
    \begin{center}
      \begin{tabular}{ccclccc|l}
           & A  &    & $\quad$ &    & E   &    & Umformungen          \\ \hline
        -1 & 3  & 2  &          & 1  & 0   & 0  & $Z_1 = Z_1 + Z_3$   \\
        -2 & 4  & 6  &          & 0  & 1   & 0  & $Z_2 = Z_2 + 2\cdot Z_3$  \\
        1  & -2 & 2  &          & 0  & 0   & 1  & $Z_3 = Z_3 - Z_1$   \\ \hline
        0  & 1  & 4  &          & 1  & 0   & 1  & $Z_1 = 5\cdot Z_1 - 2\cdot Z_2$ \\
        0  & 0  & 10 &          & 0  & 1   & 2  &             \\
        1  & -2 & 2  &          & 0  & 0   & 1  & $Z_3 = Z_3 + 2\cdot Z_1$  \\ \hline
        0  & 5  & 0  &          & 5  & -2  & 1  & $Z_1 = 2\cdot Z_1$           \\
        0  & 0  & 10 &          & 0  & 1   & 2  &             \\
        1  & 0  & 10 &          & 2  & 0   & 3  & $Z_3 = (Z_3 - Z_2)\cdot 10$            \\ \hline
        0  & 10 & 0  &          & 10 & -4  & 2  & tausche $Z_1$ mit $Z_2$       \\
        0  & 0  & 10 &          & 0  & 1   & 2  & tausche $Z_2$ mit $Z_3$      \\
        10 & 0  & 0  &          & 20 & -10 & 10 & tausche $Z_3$ mit $Z_1$           \\ \hline
        10 & 0  & 0  &          & 20 & -10 & 10 &             \\
        0  & 10 & 0  &          & 10 & -4  & 2  &             \\
        0  & 0  & 10 &          & 0  & 1   & 2  &            
      \end{tabular}
    \end{center}
\vspace*{0.5cm}
    \[ 
      A^{-1} = \frac{1}{10}\cdot
      \begin{pmatrix}
        20 & -10 & 10 \\
        10 & -4 & 2 \\
        0 & 1 & 2
      \end{pmatrix}
      = 
      \begin{pmatrix}
        2 & -1 & 1 \\
        1 & -0.4 & 0.2 \\
        0 & 0.1 & 0.2
      \end{pmatrix}
    \]
  \end{solution}


\end{document}