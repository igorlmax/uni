\documentclass[a4paper, 12pt, margins=3cm]{homework}
\usepackage{tikz}

\usepackage{graphicx}
\usepackage{dsfont}
\usepackage{microtype}
\usepackage{mathrsfs}
\usepackage[ngerman]{babel}
\usepackage{csquotes}
\usepackage[T1]{fontenc}
\usepackage{lmodern}
\usepackage{wasysym}

\setlength{\parindent}{0pt}

\newcommand{\R}{\mathbb{R}}
\newcommand{\N}{\mathbb{N}}
\newcommand{\Z}{\mathbb{Z}}
\newcommand{\Q}{\mathbb{Q}}
\newcommand{\C}{\mathbb{C}}

\name{Tobias Eidelpes}
\course{Algebra und Diskrete Mathematik}
\term{2015WS}
\hwnum{9}
\hwtype{Übungsblatt}
\problemtitle{Aufgabe}
\solutiontitle{Lösung}

\begin{document}
  \begin{center}
    \textsc{Beispiele 326, 348, 356, 360, 364, 375, 395}
  \end{center}


% ERLEDIGT 326
  \problemnumber{326}
  \begin{problem}
    Gegeben seien die folgenden Permutationen der $S_8$:
    \[ \pi = (13746),\quad \rho = (143652),\quad \text{und } \sigma = 
        \begin{pmatrix}
            1 & 2 & 3 & 4 & 5 & 6 & 7 & 8 \\
            5 & 4 & 2 & 1 & 8 & 7 & 6 & 3 \\
        \end{pmatrix}
    \]
    Berechnen Sie $\pi\rho^{-1}\sigma^2$ und $\pi^2\rho\sigma^{-2}$ sowie deren
    Zyklendarstellungen und Vorzeichen.
  \end{problem}
  \begin{solution}
    \[ \pi = (13746), \quad \rho = (143652),\quad \sigma = (24)(35)(67),\quad \rho^{-1} = (125634), \quad \sigma^2 = (1) = id \]
    \[ \pi^2 = (13746)(13746) = (17634) \]

    \[ \pi\rho^{-1}\sigma^2 = (13746)(125634)(1) = (13746)(125634) = (125)(6743) \]
    \[ \pi^2\rho\sigma^{-2} = (17634)(143652)= (6527) \]

    Vorzeichen:
    \[ sign(\pi\rho^{-1}\sigma^2) = -1 \]
    \[ sign(\pi^2\rho\sigma^{-2}) = -1 \]
  \end{solution}


% ERLEDIGT 348
  \problemnumber{348}
  \begin{problem}
    Untersuchen Sie, ob die Menge $M$ mit der Operation $\circ$ ein Gruppoid,
    eine Halbgruppe, ein Monoid bzw. eine Gruppe ist:
    \[ M = \Q\setminus \{1\},\, a\circ b = a+b-ab \]
  \end{problem}
  \begin{solution}
    Eine algebraische Struktur kann folgende Eigenschaften annehmen:
    \begin{enumerate}
      \item \textbf{Abgeschlossenheit:} $G\times G = G$ für $a,b\in G \rightarrow a\circ b\in G$.
      \item \textbf{Assoziativgesetz:} $a\circ (b\circ c) = (a\circ b)\circ c$ für alle $a,b,c\in G$.
      \item \textbf{Neutrales Element:} $\exists e\in G\; |\; \forall a\in G: a\circ e = e\circ a = a$.
      \item \textbf{Inverses Element:} $\forall a\in G\; |\; \exists a'\in G: a\circ a' = a'\circ a = e$.
      \item \textbf{Kommutativgesetz:} $a\circ b = b\circ a$ für alle $a,b\in G$.
    \end{enumerate}
    1. Eigenschaft: \emph{Gruppoid}.\\
    1. und 2. Eigenschaft: \emph{Halbgruppe}.\\
    1. 2. und 3. Eigenschaft: \emph{Monoid}.\\
    1. 2. 3. und 4. Eigenschaft: \emph{Gruppe}.\\
    1. 2. 3. 4. und 5. Eigenschaft: \emph{Abelsche Gruppe}.

    \subsubsection*{Abgeschlossenheit}
      Kann das Ergebnis 1 ergeben?
      \[ a+b-ab = 1 \]
      \[ a-ab=1-b \]
      \[ a\cdot (1-b) = 1-b \]
      \[a = \frac{1-b}{1-b} \]
      \[ a=1,\quad b=0\]
      Dasselbe mit $b$ ergibt
      \[ b=1,\quad a=0 \]
      Das heißt das Ergebnis der Operation ist nur dann 1, wenn entweder $a$ oder $b$
      eins sind, das ist aber von vornherein ausgeschlossen. Daraus folgt, dass 
      die Operation \textbf{abgeschlossen} ist.

    \subsubsection*{Assoziativität}
      Es muss gelten:
        \[ a\circ (b\circ c) = (a\circ b)\circ c \]
      Linke Seite:
        \[ a\circ (b+c-bc) = a+b+c-bc-ab-ac+abc \]
      Rechte Seite:
        \[ (a+b-ab)\circ c = a+b-ab+c-ac-bc+abc \]
      Beide Seiten sind gleich, daher ist die \textbf{Assoziativität} gegeben.

    \subsubsection*{Neutrales Element (Einheitselement)}
      Wir prüfen:
        \[ a\circ e = a+e-ae = a \]
      Daraus folgt, dass das neutrale Element 0 sein muss, also ein Einheitselement
      existiert.

    \subsubsection*{Inverses Element}
      Bei der Operation $a\circ a' = a'\circ a$ soll das Einheitselement $e$, also
      0 herauskommen.
      Daraus folgt:
        \[ a+a'-a'\cdot a = 0 \]
        \[ a+a'\cdot (1-a) = 0 \]
        \[ a'\cdot (1-a) = -a \]
        \[ a' = - \frac{a}{1-a} \]
      Das ist das inverse Element, wobei $a'$ nicht 1 sein darf, sonst folgt die
      Division durch Null, was aber sowieso ausgeschlossen ist durch $M = \Q\backslash \{1\}$.

    \subsubsection*{Kommutativität}
      Es muss gelten: $a\circ b = b\circ a$
        \[ a\circ b = a+b-ab \]
        \[ b\circ a = b+a-ba \]
      Daraus folgt, dass die Kommutativität gegeben ist.

    \subsubsection*{Schlussfolgerung}
      Die Operation ist eine \emph{Abelsche Gruppe}.
  \end{solution}



% ERLEDIGT 356
  \problemnumber{356}
  \begin{problem}
    Man ergänze die folgende Operationstafel so, dass $\langle G=\{a,b,c,d\},*\rangle$
    eine Gruppe ist.

    \begin{center}
      \begin{tabular}{c|cccc}
         $*$ & $a$ & $b$ & $c$ & $d$ \\ \hline
         $a$ & $a$ & $b$ & $c$ & $d$ \\
         $b$ & $b$ & $c$ & $d$ & $a$ \\
         $c$ & $c$ & $d$ & $a$ & $b$ \\
         $d$ & $d$ & $a$ & $b$ & $c$
      \end{tabular}
    \end{center}
  \end{problem}
  \begin{solution}
    
  \end{solution}


% AUSSTÄNDIG 360
  \problemnumber{360}
  \begin{problem}
    Man zeige: Eine nichtleere Teilmenge $U$ einer Gruppe $G$ (mit neutralem Element
    $e$) ist genau dann Untergruppe von $G$, wenn \\
    \begin{parts}
      \label{360.a}\part $a,b\in U \Rightarrow ab\in U$
      \label{360.b}\part $e\in U$
      \label{360.c}\part $a\in U \Rightarrow a^{-1}\in U$\\
    \end{parts}
    für alle $a,b\in G$ erfüllt ist. Dies ist genau dann der Fall, wenn
    $a,b\in U \Rightarrow ab^{-1}\in U$.
  \end{problem}
  \begin{solution}
    
  \end{solution}

\newpage

% ERLEDIGT 364
  \problemnumber{364}
  \begin{problem}
    Man bestimme alle Untergruppen einer zyklischen Gruppe der Ordnung 6, d.h.,
    von $G = \{e,a,a^2,a^3,a^4,a^5\}$.
  \end{problem}
  \begin{solution}\hfill
    \subsubsection*{Zyklische Gruppe}
      Eine zyklische Gruppe ist eine Gruppe, die von einem einzelnen Element $a$
      erzeugt wird. Sie besteht nur aus Potenzen des Erzeugers $a$.
    \subsubsection*{Untergruppe}
      Eine Untergruppe ist eine nichtleere Teilmenge einer Gruppe und ist selbst
      eine Gruppe (assoziativ, neutrales Element, Inverse zu jedem Element).
    \subsubsection*{Kleiner Fermat'scher Satz}
      Für jedes Element $a\in G$ einer endlichen Gruppe $(G,\circ)$ gilt $a^{|G|} = e$.
      Auf dieses Beispiel angewandt heißt das: $a^6 = e$.
    \subsubsection*{Lösung}
      Zwei Untergruppen haben wir bereits: 
        \[ U_1 = \{e\} \]
      und
        \[ U_2 = \{e,a,a^2,a^3,a^4,a^5\} \]
      Die anderen Untergruppen findet man, indem man schaut, wann die Gruppe ein
      neutrales Element besitzt und jedes Element ein Inverses hat. Die Assoziativität
      erbt die Untergruppe automatisch von der gegebenen Gruppe. Da zyklische Gruppen
      wie Restklassen gesehen werden können, gilt hier $a^6 = e$.

      Die dritte Untergruppe ist 
        \[ U_3 = \{e,a^3\} \]
      da $a^3\circ a^3 = a^6 = e$ ist. $a^3$ ist also das inverse Element zu sich selbst.
      $\{e,a^4\}$ und $\{e,a^5\}$ sind keine Untergruppen, da ein inverses Element fehlt.

      Als nächstes gilt es Untergruppen mit drei Elementen zu finden. Dabei kann 
      man sich das ganze Vereinfachen, indem man sich überlegt, dass die Exponenten
      der Elemente addiert die Ordnung, also 6, ergeben müssen. Daraus folgt:
        \[ U_4 = \{e, a^2, a^4\} \] 
        \[ a^2\circ a^4 = a^6 = e \text{ und } a^2\circ a^2 = a^4 \text{ bzw. } a^4\circ a^4 = a^8 = a^6\circ a^2 = e\circ a^2 = a^2 \]

  \end{solution}


% ERLEDIGT 375
  \problemnumber{375}
  \begin{problem}
    Man zeige, dass die von $\bar{3}$ erzeugte Untergruppe $U$ von $\langle \Z_{12}, +\rangle$
    ein Normalteiler von $\langle \Z_{12}, +\rangle$ ist und bestimme die Gruppentafel
    der Faktorgruppe $\Z_{12}/U$.
  \end{problem}
  \begin{solution}\hfill
    Eine Untergruppe $N\leq G$ heißt Normalteiler, wenn stets LNK=RNK gilt, d.h.
    $aN = Na, \forall a\in G$. Außerdem gilt:\\

    In abelschen Gruppen ist jede Untergruppe Normalteiler, somit lässt sich dort
    nach jeder Untergruppe die Faktorgruppe bilden.\\

    \[ U = \{0,3,6,9\} \]
    $U$ ist Normalteiler, weil $\langle \Z_{12}, +\rangle$ kommutativ ist und daher
    die Linksnebenklassen mit den Rechtsnebenklassen übereinstimmen müssen.\\

    Nebenklassen:
      \[ a := 0+U = 3+U = 6+U = 9+U = \{0,3,6,9\} \]
      \[ b := 1+U = 4+U = 7+U = 10+U = \{1,4,7,10\} \]
      \[ c := 2+U = 5+U = 8+U = 11+U = \{2,5,8,11\} \]

    Gruppentafel:
      \begin{center}
        \begin{tabular}{c|ccc}
         $+$ & $a$ & $b$ & $c$ \\ \hline
         $a$ & $a$ & $b$ & $c$ \\
         $b$ & $b$ & $c$ & $a$ \\
         $c$ & $c$ & $a$ & $b$ \\        
        \end{tabular}
      \end{center}
    $a$ ist das neutrale Element und $b$ ist zu $c$ invers.
  \end{solution}


% AUSSTÄNDIG 395
  \problemnumber{395}
  \begin{problem}
    Von der Abbildung $f:(\Z_3)^2 \rightarrow (\Z_3)^4$ sei bekannt, dass $f$ ein
    Gruppenhomomorphismus bezüglich der Addition ist (die jeweils komponentenweise
    definiert sein soll), sowie dass $f(1,0)=(1,0,0,2), f(1,1)=1,2,0,1$. Man ermittle
    daraus $f(w)$ für alle $w\in (\Z_3)^2$.
  \end{problem}
  \begin{solution}
    
  \end{solution}
\end{document}
