\documentclass[a4paper, 12pt, margins=3cm]{homework}
\usepackage{tikz}

\usepackage{graphicx}
\usepackage{dsfont}
\usepackage{microtype}
\usepackage{mathrsfs}
\usepackage[ngerman]{babel}
\usepackage{csquotes}
\usepackage[T1]{fontenc}
\usepackage{lmodern}
\usepackage{wasysym}

\setlength{\parindent}{0pt}

\newcommand{\R}{\mathbb{R}}
\newcommand{\N}{\mathbb{N}}
\newcommand{\Z}{\mathbb{Z}}
\newcommand{\Q}{\mathbb{Q}}
\newcommand{\C}{\mathbb{C}}

\name{Tobias Eidelpes}
\course{Algebra und Diskrete Mathematik}
\term{2015WS}
\hwnum{8}
\hwtype{Übungsblatt}
\problemtitle{Aufgabe}
\solutiontitle{Lösung}

\begin{document}


  \begin{center}
    \textsc{Beispiele 207, 232, 331}
  \end{center}


% 207 ERLEDIGT
  \problemnumber{207}
  \begin{problem}
    Wie viele Möglichkeiten gibt es, $2n$ Punkte auf einer Geraden so oberhalb der
    Geraden paarweise zu verbinden, dass sich die Verbindungslinien nicht kreuzen?
  \end{problem}
  \begin{solution}
    Dieses Problem kann mithilfe der Catalan-Zahlen gelöst werden:
    \[ C_n = \frac{1}{n+1}\cdot \binom{2n}{n} \]
    wobei $n$ die Anzahl der Punktpaare ist und $C_n$ folglich alle Kombinationen
    diese Paare miteinander zu verbinden, ohne dass Überschneidungen stattfinden,
    darstellt. Das ganze als Rekursionsformel:
    \[ C_{n} = \sum_{k=1}^{n}{C_{k-1}C_{n-k}} \] 
  \end{solution}


% 232 AUSSTÄNDIG
  \problemnumber{232}
  \begin{problem}
    Lösen Sie die Rekursion mit der Ansatzmethode:
    \[ a_n = 3a_{n-1} + 3^{n-1}\qquad (n\geq 1), a_0 = 2 \]
  \end{problem}
  \begin{solution}
    
  \end{solution}


% 331 AUSSTÄNDIG
  \problemnumber{331}
  \begin{problem}
    Gegeben seien die folgenden zweistelligen partiellen Operationen $\bullet$ in
    der Menge $M$. Man untersuche, in welchem Fall eine Operation in $M$ vorliegt.
    Welche der Operationen sind assoziativ, welche kommutativ?
    \begin{parts}
      \part \label{331.a} $ M = \{ -1, 0, 1 \},\, \bullet \text{ gewöhnliche Addition bzw. Multiplikation} $
      \part \label{331.b} $ M = \N,\, a\bullet b = 2^{ab} $
      \part \label{331.c} $ M = \Q,\, a\bullet b = ab + 1 $
      \part \label{331.d} $ M = \R,\, a\bullet b = |a + b| $
      \part \label{331.e} $ M \neq \emptyset,\, a\bullet b = a $
    \end{parts}
  \end{problem}
  \begin{solution} \hfill

    \begin{enumerate}[label=(\alph*)]\itemsep0pt
      \item \hfill
        \begin{description}\itemsep0pt
          \item[Addition] Keine Operation in $M$, weil $1+1=2\notin M$ (nicht abgeschlossen, kommutativ, assoziativ)
          \item[Multiplikation] Operation in $M$, weil alle Multiplikationen Elemente aus $M$ sind. (abgeschlossen, kommutativ, assoziativ)
        \end{description}
      \item
      \item \hfill
        \begin{description}\itemsep0pt
          \item[Addition] Operation in $\Q$ \\
                Assoziativ: nein
                            \[ (a\bullet b)\bullet c = (ab+1)\bullet c = (ab+1) + c+1 = ab+c+2 \]
                            \[ a\bullet (b\bullet c) = a\bullet (bc+1) = a + 1 + (bc+1) = a+bc+2 \]
                            \[ \Longrightarrow \text{nicht gleich, daher nicht assoziativ} \]
                Kommutativ: ja
          \item[Multiplikation] Operation in $\Q$ \\
                Assoziativ: nein
                            \[ (a\bullet b)\bullet c = (ab+1)\bullet c = (ab+1)\cdot c+1 = abc + c + 1 \]
                            \[ a\bullet (b\bullet c) = a\bullet (bc+1) = a\cdot (bc+1)+1 = abc + a + 1 \]
                            \[ \Longrightarrow \text{nicht gleich, daher nicht assoziativ} \]
                Kommutativ: ja
        \end{description}
      \item \hfill

        abgeschlossen ja, kommutativ ja \\
        assoziativ:
        \[ (a\bullet b)\bullet c = ||a+b|+c| = ||4-3|-5| = 4 \]
        \[ a\bullet (b\bullet c) = |a+|b+c|| = |4+|-3-5|| = 12 \]
        \[ \Longrightarrow \text{nicht gleich, daher nicht assoziativ} \]

      \item \hfill 

        abgeschlossen ja, kommutativ nein, denn $a\bullet b = a$, aber $b\bullet a = b$
        assoziativ:
        \[ (a\bullet b)\bullet c = a\bullet c = a \]
        \[ a\bullet (b\bullet c) = a\bullet b = a \]
        \[ \Longrightarrow \text{gleich, daher assoziativ} \]
    \end{enumerate}
  \end{solution}


\end{document}