\documentclass[a4paper, margins=3cm, newpage]{homework}

\usepackage{polynom}

\polyset{%
  style=C,
  delims={\big(}{\big)},
  div=:
}

\name{Tobias Eidelpes}
\course{Technische Grundlagen der Informatik}
\term{2015WS}
\hwnum{1}
\hwtype{Übungsblatt}
\problemtitle{Aufgabe}
\solutiontitle{Lösung}

\begin{document}

\begin{problem}
Gegeben sind die folgenden Dezimalzahlen:
\begin{align*}
	A &= (108.1625)_{10} \\
	B &= (-25.40625)_{10}
\end{align*}
Wandeln Sie die Zahlen \(A\) und \(B\) direkt in die nachfolgend angegebenen Zahlensysteme um. Geben Sie das Ergebnis auf \emph{n} Nachkommastellen genau an. Runden Sie Ihr Ergebnis durch \emph{round to nearest} (Optimale Rundung) auf \emph{n} Nachkommastellen. Falls es zwei nächstliegende Zahlen gibt, verwenden Sie \emph{round to even}.
Geben Sie Ihre Berechnungen sowie die verwendete Rundungsmethode an!
\begin{parts}
\part
\label{1.a}
Binärsystem, \(n=4\)
\part
\label{1.b}
Hexadezimalsystem, \(n=2\)
\part
\label{1.c}
Ternäres Zahlensystem, \(n=4\)
\end{parts}
\end{problem}

\begin{solution}
\ref{1.a} für \(A\) \\
\begin{minipage}{0.5\textwidth}
\begin{align*}
	108 \text{ mod } 2 \quad &0 \\
	44 \text{ mod } 2 \quad &0 \\
	27 \text{ mod } 2 \quad &1 \\
	13 \text{ mod } 2 \quad &1 \\
	6 \text{ mod } 2 \quad &0 \\
	3 \text{ mod } 2 \quad &1 \\
	1 \text{ mod } 2 \quad &1 \\
\end{align*}
\end{minipage}
\begin{minipage}{0.5\textwidth}
\begin{align*}
	0.1625 \cdot 2 = 0.3250 \quad & 0 \\
	0.3250 \cdot 2 = 0.65 \quad & 0 \\
	0.65 \cdot 2 = 1.3 \quad & 1 \\
	0.3 \cdot 2 = 0.6 \quad & 0 \\
	0.6 \cdot 2 = 1.2 \quad & 1 \\
	0.2 \cdot 2 = 0.4 \quad & 0 \\
	0.4 \cdot 2 = 0.8 \quad & 0 \\
	0.8 \cdot 2 = 1.6 \quad & 1 \\
\end{align*}
\end{minipage}

\[(108.1625)_{10} = (1101100.0011)_2\]
Rundungsmethode: \emph{round to nearest} \\

\ref{1.a} für \(B\) \\
\begin{minipage}{0.5\textwidth}
\begin{align*}
	25 \text{ mod } 2 \quad &1 \\
	12 \text{ mod } 2 \quad &0 \\
	6 \text{ mod } 2 \quad &0 \\
	3 \text{ mod } 2 \quad &1 \\
	1 \text{ mod } 2 \quad &1 \\
\end{align*}
\end{minipage}
\begin{minipage}{0.5\textwidth}
\begin{align*}
	0.40625 \cdot 2 = 0.8125 \quad & 0 \\
	0.8125 \cdot 2 = 1.625 \quad & 1 \\
	0.625 \cdot 2 = 1.25 \quad & 1 \\
	0.25 \cdot 2 = 0.5 \quad & 0 \\
	0.5 \cdot 2 = 1 \quad & 1 \\
\end{align*}
\end{minipage}

\[(-25.40625)_{10} = (-11001.0110)_2\]
Rundungsmethode: \emph{round to even} 

\ref{1.b} für \(A\) \\
\begin{minipage}{0.5\textwidth}
\begin{align*}
	108 \text{ mod } 16 \quad &C \\
	6 \text{ mod } 16 \quad &6 \\
\end{align*}
\end{minipage}
\begin{minipage}{0.5\textwidth}
\begin{align*}
	0.1625 \cdot 16 = 2.6 \quad & 2 \\
	0.6 \cdot 16 = 9.6 \quad & 9 \\
	0.6 \cdot 16 = 9.6 \quad & 9
\end{align*}
\end{minipage}

\[(108.1625)_{10} = (6C.2A)_{16}\]
Rundungsmethode: \emph{round to nearest} \\

\ref{1.b} für \(B\) \\
\begin{minipage}{0.5\textwidth}
\begin{align*}
	25 \text{ mod } 16 \quad &9 \\
	1 \text{ mod } 16 \quad &1 
\end{align*}
\end{minipage}
\begin{minipage}{0.5\textwidth}
\begin{align*}
	0.40625 \cdot 16 = 6.5 \quad & 6 \\
	0.5 \cdot 16 = 8 \quad & 8 
\end{align*}
\end{minipage}

\[(-25.40625)_{10} = (-19.68)_{16}\]
Rundungsmethode: \emph{exakt}

\ref{1.c} für \(A\) \\
\begin{minipage}{0.5\textwidth}
\begin{align*}
	108 \text{ mod } 3 \quad &0 \\
	36 \text{ mod } 3 \quad &0 \\ 
	12 \text{ mod } 3 \quad &0 \\
	4 \text{ mod } 3 \quad &1 \\
	1 \text{ mod } 3 \quad &1
\end{align*}
\end{minipage}
\begin{minipage}{0.5\textwidth}
\begin{align*}
	0.1625 \cdot 3 = 0.4875 \quad & 0 \\
	0.4875 \cdot 3 = 1.4625 \quad & 1 \\ 
	0.4625 \cdot 3 = 1.3875 \quad & 1 \\ 
	0.3875 \cdot 3 = 1.1625 \quad & 1 \\ 
	0.1625 \cdot 3 = 0.4875 \quad & 0
\end{align*}
\end{minipage}

\[(108.1625)_{10} = (11000.0111)_3\]
Rundungsmethode: \emph{round to nearest} \\

\ref{1.c} für \(B\) \\
\begin{minipage}{0.5\textwidth}
\begin{align*}
	25 \text{ mod } 3 \quad &1 \\
	8 \text{ mod } 3 \quad &2 \\ 
	2 \text{ mod } 3 \quad &2 \\
\end{align*}
\end{minipage}
\begin{minipage}{0.5\textwidth}
\begin{align*}
	0.40625 \cdot 3 = 1.21875 \quad & 1 \\
	0.21875 \cdot 3 = 0.65625 \quad & 0 \\ 
	0.65625 \cdot 3 = 1.96875\quad & 1 \\ 
	0.96875 \cdot 3 = 2.90625 \quad & 2 \\ 
	0.90625 \cdot 3 = 2.71875 \quad & 2
\end{align*}
\end{minipage}

\[(-25.40625)_{10} = (-221.1020)_3\]
Rundungsmethode: \emph{round to nearest}

\end{solution}

\begin{problem}
Führen Sie die folgenden Umwandlungen \emph{ohne} Umweg über das Dezimalsystem durch!
\begin{parts}
\part
\label{2.a}
Wandeln Sie die Hexadezimalzahl \((67.EC)_{16}\) in eine Binärzahl um.
\part
\label{2.b}
Wandeln Sie die Binärzahl \((1101100.11101)_2\) in eine Quarterärzahl um.
\part
\label{2.c}
Wandeln Sie die Zahl \((824.75)_9\) in eine ternäre Zahl um.
\end{parts}
\end{problem}

\begin{solution}
\ref{2.a}
\begin{align*}
	&6& &7& &.& &E& &C& \\
	&0110& &0111& &.& &1110& &1100&
\end{align*}
\[(01100111.11101100)_2 = (67.EC)_{16}\] \\

\ref{2.b}
\begin{align*}
	&01& &10& &11& &00& &.& &11& &10& &10& &00& \\
	&1& &2& &3& &0& &.& &3& &2& &2& &0&
\end{align*}
\[(1101100.11101)_2 = (1230.322)_{4}\] \\

\ref{2.c}
\begin{align*}
	&8& &2& &4& &.& &7& &5& \\
	&22& &02& &11& &.& &21& &12&
\end{align*}
\[(824.75)_9 = (220211.2112)_3\]

\end{solution}

\begin{problem}
Es sind die folgenden Binärzahlen gegeben:
\begin{align*}
	A &= (10011.01)_2 \\
	B &= (1110.11)_2 \\
	C &= (100)_2 \\
	D &= (-101.1)_2
\end{align*}
Führen Sie mit diesen Zahlen die folgenden arithmetischen Operationen binär(!) durch. Berechnen Sie die Ergebnisse exakt und geben Sie Ihren Rechenweg an!
\begin{parts}
\part
\label{3.a}
Addition: \(A + B\)
\part
\label{3.b}
Subtraktion: \(A - B\)
\part
\label{3.c}
Division: \(B\div C\)
\end{parts}
\end{problem}

\begin{solution}
\ref{3.a}
\begin{center}
\begin{tabular*}{0.17\textwidth}{c@{\,}c@{\,}c@{\,}c@{\,}c@{\,}c@{\,}c@{\,}c@{\,}c@{\,}c@{\,}c@{\,}}
	 & &  & 1 & 0 & 0 & 1 & 1 & . & 0 & 1 \\
	+& & &   & 1 & 1 & 1 & 0 & . & 1 & 1 \\
	\hline
	& & 1 & 0 & 0 & 0 & 1 & 0 & . & 0 & 0 \\
\end{tabular*}
\end{center}
\[A+B = (100010)_2\]
 
\ref{3.b}
\begin{center}
\begin{tabular*}{0.17\textwidth}{c@{\,}c@{\,}c@{\,}c@{\,}c@{\,}c@{\,}c@{\,}c@{\,}c@{\,}c@{\,}c@{\,}}
	 & &  & 1 & 0 & 0 & 1 & 1 & . & 0 & 1 \\
	-& & &   & 1 & 1 & 1 & 0 & . & 1 & 1 \\
	\hline
	& &  & 0 & 0 & 1 & 0 & 0 & . & 1 & 0 \\
\end{tabular*}
\end{center}
\[A-B = (100.1)_2\]

\ref{3.c}
\begin{center}
\begin{tabular}{c@{\,}c@{\,}c@{\,}c@{\,}c@{\,}c@{\,}c@{\,}c@{\,}c@{\,}c@{\,}c@{\,}c@{\,}c@{\,}c@{\,}c@{\,}c@{\,}c@{\,}c@{\,}c@{\,}c@{\,}}
	  & 1 & 1 & 1 & 0. & 1 & 1 & $\div$ & 1 & 0 & 0 & = & 1 & 1 & . & 1 & 0 & 1 & 1 \\
	- & 1 & 0 & 0 &&&&&&&&&&&&&&&& \\
	\hline
	  & 0 & 1 & 1 & 0&&&&&&&&&&&&&& \\
	- &   & 1 & 0 & 0&&&&&&&&&&&&&& \\
	\hline
	  &   & 0 & 1 & 0& 1 &&&&&&&&&&&&& \\
	- &   & 0 & 1 & 0 & 0 &  &&&&&&&&&&&&& \\
	\hline
	  &   &   & 0 & 0 & 1 & 1 &&&&&&&&&&&&& \\
	- &   &   & 0 & 0 & 0 & 0 &&&&&&&&&&&&& \\
	\hline
	  &   &   &   &   & 1 & 1 & 0 &&&&&&&&&&&& \\
	- &   &   &   &   & 1 & 0 & 0 &&&&&&&&&&&& \\
	\hline 
	  &   &   &   &   & 0 & 1 & 0 & 0 &&&&&&&&&&& \\
	- &   &   &   &   &   & 1 & 0 & 0 &&&&&&&&&&& \\
	\hline
	 &   &   &   &   &  &  &  & 0 &&&&&&&&&&&
\end{tabular}
\end{center}
\[A\div B = (11.1011)_2\]

\end{solution}

\begin{problem}
Es sind folgende Zahlen gegeben:
\begin{align*}
	A &= (2A3)_{16} \\
	B &= (-102)_4 \\
	C &= (0)_2
\end{align*}
Geben Sie die Zahlen \(A\), \(B\) und \(C\) als 12 Bit lange Maschinenwörter in den nachfolgenden Zahlendarstellungen
jeweils in binärer – z.B. 010 1110 0111 – und in hexadezimaler – z.B. $2E7$ – \underline{Notation} an. Falls es in einer
Zahlendarstellung für dieselbe Zahl unterschiedliche Darstellungen gibt, geben Sie alle an!
\begin{parts}
\part
\label{4.a}
Vorzeichen und Betrag
\part
\label{4.b}
Einerkomplementdarstellung
\part
\label{4.c}
Zweierkomplementdarstellung
\part
\label{4.d}
Exzessdarstellung (Exzess \(= 2^7 - 1\))
\end{parts}
\end{problem}

\begin{solution}
\ref{4.a} für \(A\) \\
\[(2A3)_{16} = (0010\;1010\;0011)_2\]
\begin{center}
Vorzeichen und Betrag: \(0 \;|\; 010\;1010\;0011 = (2A3)_{16}\)
\end{center}

\ref{4.a} für \(B\) \\
\[(-102)_4 = (-0000\;0001\;0010)_2 = (-012)_{16}\]
\begin{center}
Vorzeichen und Betrag: \(1 \;|\; 000\;0001\;0010 = (812)_{16}\) 
\end{center}

\ref{4.a} für \(C\) \\
\[(0)_2 = (0)_{16}\]
\begin{align*}
	\text{Vorzeichen und Betrag: } &0 \;|\; 000\;000\;000 = (0)_{16} \text{ und} \\
								  &1 \;|\; 000\;000\;000 = (800)_{16} \\
\end{align*}

\ref{4.b} für \(A\) \\
\[(2A3)_{16} = (0010\;1010\;0011)_2\]
\begin{center}
	Einerkomplement: \((0010\;1010\;0011)_2 = (2A3)_{16}\)
\end{center}

\ref{4.b} für \(B\) \\
\[(-102)_4 = (-0000\;0001\;0010)_2 = (-012)_{16}\]
\begin{center}
	Einerkomplement: \((1111\;1110\;1101)_2 = (7ED)_{16}\)
\end{center}

\ref{4.b} für \(C\) \\
\[(0)_2 = (0)_{16}\]
\begin{align*}
	\text{Einerkomplement: } & (0000\;0000\;0000)_2 = (0)_{16} \text{ und} \\
							 & (1111\;1111\;1111)_2 = (FFF)_{16}
\end{align*}

\ref{4.c} für \(A\) \\
\[(2A3)_{16} = (0010\;1010\;0011)_2\]
\begin{center}
	Zweierkomplement: \((0010\;1010\;0011)_2 = (2A3)_{16}\)
\end{center}

\ref{4.c} für \(B\) \\
\[(-102)_4 = (-0000\;0001\;0010)_2 = (-012)_{16}\]
\begin{center}
	Zweierkomplement: \((1111\;1110\;1110)_2 = (FEE)_{16}\)
\end{center}

\ref{4.c} für \(C\) \\
\[(0)_2 = (0)_{16}\]
\begin{center}
	Zweierkomplement: \((0000\;0000\;0000)_2 = (0)_{16}\)
\end{center}

\ref{4.d} für \(A\) \\
\[(2A3)_{16} = (0010\;1010\;0011)_2\]
\begin{center}
	Exzessdarstellung mit Exzess \(= 2^7 -1\): 
	\begin{tabular}{c@{\,}c@{\,}c@{\,}c@{\,}c@{\,}c@{\,}c@{\,}c@{\,}c@{\,}c@{\,}c@{\,}c@{\,}c@{\,}}
		& 0 & 0 & 1 & 0 & 1 & 0 & 1 & 0 & 0 & 0 & 1 & 1 \\
	+	& 0 & 0 & 0 & 0 & 0 & 1 & 1 & 1 & 1 & 1 & 1 & 1 \\
	\hline
		& 0 & 0 & 1 & 1 & 0 & 0 & 1 & 0 & 0 & 0 & 1 & 0
	\end{tabular}
	\[(0011\;0010\;0010)_2 = (322)_{16}\]
\end{center}

\ref{4.d} für \(B\)
\[(-102)_4 = (-0000\;0001\;0010)_2 = (-012)_{16}\]
\begin{center}
	Exzessdarstellung mit Exzess \(= 2^7 -1\): 
	\begin{tabular}{c@{\,}c@{\,}c@{\,}c@{\,}c@{\,}c@{\,}c@{\,}c@{\,}c@{\,}c@{\,}c@{\,}c@{\,}c@{\,}}
	-	& 0 & 0 & 0 & 0 & 0 & 0 & 0 & 1 & 0 & 0 & 1 & 0 \\
	+	& 0 & 0 & 0 & 0 & 0 & 1 & 1 & 1 & 1 & 1 & 1 & 1 \\
	\hline
		& 0 & 0 & 0 & 0 & 0 & 1 & 1 & 0 & 1 & 1 & 0 & 1
	\end{tabular} \\
	(ist dasselbe wie Exzess weniger \(B\))
\end{center}
\[(0000\;0110\:1101) = (06D)_{16}\]

\ref{4.d} für \(C\)
\[(0)_2 = (0)_{16}\]
\begin{center}
	Exzessdarstellung mit Exzess \(= 2^7 -1\): 
	\begin{tabular}{c@{\,}c@{\,}c@{\,}c@{\,}c@{\,}c@{\,}c@{\,}c@{\,}c@{\,}c@{\,}c@{\,}c@{\,}c@{\,}}
		& 0 & 0 & 0 & 0 & 0 & 0 & 0 & 0 & 0 & 0 & 0 & 0 \\
	+	& 0 & 0 & 0 & 0 & 0 & 1 & 1 & 1 & 1 & 1 & 1 & 1 \\
	\hline
		& 0 & 0 & 0 & 0 & 0 & 1 & 1 & 1 & 1 & 1 & 1 & 1
	\end{tabular} \\
\end{center}
\[(0000\;0111\;1111)_2 = (07F)_{16}\]
\end{solution}

\begin{problem}
	Folgende Bitmuster sind gegeben: \(Z_1 = (00001100)_2\) und \(Z_2 = (10011011)_2\).
	Interpretieren Sie $Z_1$ und $Z_2$ als Binärzahlen, die beide jeweils in einer der nachfolgend angegebenen Darstellungen $a)$ bis $c)$ codiert sind. Führen Sie damit die Berechnung $-(Z_1 + Z_2)$
	mit einer Maschinenwortlänge von 8 Bit binär durch und geben Sie Zwischenschritte an. Geben Sie das Ergebnis der Berechnung auch als decodierte Dezimalzahl an! 
	\begin{parts}
		\part
		\label{5.a}
		Darstellung durch Vorzeichen und Betrag
		\part
		\label{5.b}
		Zweierkomplementdarstellung
		\part
		\label{5.c}
		Exzessdarstellung mit Exzess = $(10000001)_2$
	\end{parts}
\end{problem}
\begin{solution}
	\ref{5.a}
	\[ Z_1 = (0\,|\, 0001100)_2 = (+12)_{10} \]
	\[ Z_2 = (1\,|\, 0011011)_2 = (-27)_{10} \]
	\begin{center}
		$(Z_1 + Z_2) = $
		\begin{tabular}{c@{\,}c@{\,}c@{\,}c@{\,}c@{\,}c@{\,}c@{\,}c@{\,}c@{\,}}
		  & 0 & 0 & 0 & 0 & 1 & 1 & 0 & 0 \\
		- & 0 & 0 & 0 & 1 & 1 & 0 & 1 & 1 \\ \hline
		  & 0 & 0 & 0 & 0 & 1 & 1 & 1 & 1
		\end{tabular}
	\end{center}
	\[-(Z_1+Z_2) = (-0000\;1111)_2 = (1\;|\; 0001111)_2\]

	\ref{5.b}
	\[Z_1 = (1111\;0100)_2 = (-244)_{10}\]
	\[Z_2 = (0110\;0101)_2 = (101)_{10}\]
	\begin{center}
		\((Z_1+Z_2 = )\)
		\begin{tabular}{c@{\,}c@{\,}c@{\,}c@{\,}c@{\,}c@{\,}c@{\,}c@{\,}c@{\,}}
		  & 0 & 0 & 0 & 0 & 1 & 1 & 0 & 0 \\
		+ & 1 & 0 & 0 & 1 & 1 & 0 & 1 & 1 \\ \hline
		  & 1 & 0 & 1 & 0 & 0 & 1 & 1 & 1
		\end{tabular}
	\end{center}
	\[-(Z_1 + Z_2) = (0101\;1001)_2\]

	\ref{5.c}
	\[Z_1 = (00001100)_2 - (01111111)_2 = (-01110011)_2 = (-115)_{10}\]
	\[Z_2 = (10001101)_2 - (01111111)_2 = (00001110)_2 = (14)_{10}\]
	\begin{center}
		\((Z_1+Z_2)=\)
		\begin{tabular}{c@{\,}c@{\,}c@{\,}c@{\,}c@{\,}c@{\,}c@{\,}c@{\,}c@{\,}}
		  & 0 & 0 & 0 & 0 & 1 & 1 & 0 & 0 \\
		+ & 1 & 0 & 0 & 1 & 1 & 0 & 1 & 1 \\ \hline
		  & 1 & 0 & 1 & 0 & 0 & 1 & 0 & 1
		\end{tabular}
	\end{center}
	\[-(Z_1+Z_2)= (0010\;1000)_2\]
\end{solution}

\begin{problem}
	Wandeln Sie die Zahl $(1.6875)_{10}$ in eine Binärzahl mit 3 Nachkommastellen
	um - alle weiteren Nachkommastellen werden abgeschnitten.
	\begin{parts}
		\part
		\label{6.a}
		Berechnen Sie den absoluten sowie den relativen Rundungsfehler, der bei der Umrechnung
		ins Binärsystem entstanden ist.
		\part
		\label{6.b}
		Durch die Rundungsmethode werden alle reellen Zahlen aus einem Intervall $[a,b[\; \in \mathbb{R}$ auf dieselbe Binärzahl abgebildet. Geben Sie die dezimalen Werte $a,b$ für das Intervall an, in dem $(1.6875)_{10}$ liegt!
	\end{parts}
\end{problem}
\begin{solution}
	\ref{6.a}
	\[(1.6875)_{10} = (1.1011)_2 \approx (1.101)_2 = (1.625)_{10}\]
	\begin{align*}
		\text{relativer Fehler: }& 1-(1.625\div 1.6875) \approx 0.037 \approx 3.7\% \\
		\text{absoluter Fehler: }& 1.6875-1.6250 = 0.0625
	\end{align*}

	\ref{6.b}
	\[ [1.6250, 1.7499] \]
\end{solution}

\begin{problem}
	Stellen Sie die nachfolgenden Zahlen $A$ und $B$ im \emph{Single Precision}-Format
	(mit implizitem ersten Bit) der IEEE 754 Gleitpunkt-Zahlensysteme dar.
	\[ A=(0.D266)_{16} \]
	\[ B=(-310.11)_{4} \]
\end{problem}

\begin{solution}
	\[(0.D266)_{16} = (0.1101\;0010\;0110\;0110)_2\]
	\[\text{Normalisierung: } (1.101\;0010\;0110\;0110)_2 \cdot 2^{-1}\]
	\[ \text{Exponent: } (01111111) - (00000001) = (01111110) \]
	\[ \text{Vorzeichen: } (0)_2 \]
	\[ 0\;|\; 01111110\;|\; 11010010011001100000000 \]

	\[(-310.11)_4 = (-110100.0101)_2\]
	\[ \text{Normalisierung: } (1.101000101)_2 \cdot 2^5 \]
	\[ \text{Exponent: } (01111111) + (00000101) = (10000100) \]
	\[ \text{Vorzeichen: } (1)_2 \]
	\[ 1\;|\; 10000100\;|\; 11010001010000000000000 \]
\end{solution}

\end{document}