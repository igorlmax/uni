\documentclass[a4paper, 12pt, margins=2cm]{homework}
\usepackage{tikz}

\usepackage{graphicx}
\usepackage{dsfont}
\usepackage{microtype}
\usepackage{mathrsfs}
\usepackage[ngerman]{babel}
\usepackage{csquotes}
\usepackage[T1]{fontenc}
\usepackage{lmodern}
\usepackage{wasysym}
\usepackage{tabularx}
\usepackage{listings}
\usepackage{algpseudocode}
\usepackage[linesnumbered, lined, boxed, commentsnumbered]{algorithm2e}

\setlength{\parindent}{0pt}

\newcommand{\R}{\mathbb{R}}
\newcommand{\N}{\mathbb{N}}
\newcommand{\Z}{\mathbb{Z}}
\newcommand{\Q}{\mathbb{Q}}
\newcommand{\C}{\mathbb{C}}

\name{Tobias Eidelpes}
\course{Technische Grundlagen der Informatik}
\term{2015WS}
\hwnum{7}
\hwtype{Übungsblatt}
\problemtitle{Aufgabe}
\solutiontitle{Lösung}

\begin{document}


% 1. ERLEDIGT
\begin{problem}
  
\end{problem}
\begin{solution}\hfill

  \begin{enumerate}[label=(\alph*)]\itemsep0pt

    \item \hfill 
      \begin{enumerate}[label=(\arabic*)]\itemsep0pt
        \item \textbf{Direct Mapped Cache}
          \begin{description}
            \item[Adresslänge] $log_2(32 \text{ GiB} = 35\text{ bit})$
            \item[Tag] $log_2(32 \text{ GiB}) - log_2(256 \text{ KiB}) \approx 17 \text{ bit}$
            \item[Index] $256 \text{ KiB} \div 128 \text{ byte} = 2048 \text{ Blöcke} = 2^{11}$
            \item[Offset] $log_2(128 \text{ byte}) = 7 \text{ bit}$
          \end{description}
        \item \textbf{4-Way Set Associative Cache}
          \begin{description}
            \item[Adresslänge] $log_2(32 \text{ GiB} = 35\text{ bit})$
            \item[Tag] $log_2(32 \text{ GiB}) - log_2(256 \text{ KiB}) \approx 17 \text{ bit}$
            \item[Index] $256 \text{ KiB} \div 128 \text{ byte} = 2048 \text{ Blöcke} = 2^{11}$
            \item[Offset] $log_2(128 \text{ byte}) = 7 \text{ bit}$
          \end{description}
        \item \textbf{Fully Associative Cache}
          \begin{description}
            \item[Adresslänge] $log_2(32 \text{ GiB} = 35\text{ bit})$
            \item[Tag] $log_2(32 \text{ GiB}) - log_2(128 \text{ byte}) \approx 28 \text{ bit}$
            \item[Index] 0 bit
            \item[Offset] $log_2(128 \text{ byte}) = 7 \text{ bit}$
          \end{description}
      \end{enumerate}

    \item \begin{align*}
            \text{0xBADCODED} &= (1011\,1010\,1101\,1100\,0000\,1101\,1110\,1101)_2 \\
            \text{0x12345432} &= (0001\,0010\,0011\,0100\,0101\,0100\,0011\,0010)_2
          \end{align*}
      \begin{enumerate}[label=(\arabic*)]\itemsep0pt
        \item \textbf{Direct Mapped Cache}
          \begin{enumerate}[label=(\arabic*)]\itemsep0pt
            \item \hfill
              \begin{center}
                \begin{tabular}{|c|c|c|}
                  \hline
                  Tag               & Index       & Offset  \\ \hline \hline
                  00010111010110111 & 00000011011 & 1101101 \\ \hline
                  2EB7              & 1B          & 6D      \\ \hline
                \end{tabular}
              \end{center}
            \item \hfill
              \begin{center}
                \begin{tabular}{|c|c|c|}
                  \hline
                  Tag         & Index       & Offset  \\ \hline \hline
                  10010001101 & 00010101000 & 0110010 \\ \hline
                  048D        & A8          & 32      \\ \hline
                \end{tabular}
              \end{center}
          \end{enumerate}
        \item \textbf{4-Way Set Associative Cache}
          \begin{enumerate}[label=(\arabic*)]\itemsep0pt
            \item \hfill
              \begin{center}
                \begin{tabular}{|c|c|c|}
                  \hline
                  Tag                 & Index     & Offset  \\ \hline \hline
                  0001011101011011100 & 000011011 & 1101101 \\ \hline
                  BADC                & 1B        & 6D      \\ \hline
                \end{tabular}
              \end{center}
            \item \hfill
              \begin{center}
                \begin{tabular}{|c|c|c|}
                  \hline
                  Tag               & Index     & Offset  \\ \hline \hline
                  00010010001101000 & 010101000 & 0110010 \\ \hline
                  2468              & A8        & 32      \\ \hline
                \end{tabular}
              \end{center}
          \end{enumerate}
        \item \textbf{Fully Associative Cache}
          \begin{enumerate}[label=(\arabic*)]\itemsep0pt
            \item \hfill
              \begin{center}
                \begin{tabular}{|c|c|}
                  \hline
                  Tag                          & Offset  \\ \hline \hline
                  0001011101011011100000011011 & 1101101 \\ \hline
                  175B81B                      & 6D      \\ \hline
                \end{tabular}
              \end{center}
            \item \hfill
              \begin{center}
                \begin{tabular}{|c|c|c|}
                  \hline
                  Tag                    & Offset  \\ \hline \hline
                  1001000110100010101000 & 0110010 \\ \hline
                  2468A8                 & 32      \\ \hline
                \end{tabular}
              \end{center}
          \end{enumerate}
      \end{enumerate}
        
    \item \hfill
      \begin{description}
        \item[Direct Mapped Cache]\hfill 
          \[ 17 \text{ bit Tag} + 1 \text{ Valid} + 1 \text{ Dirty} = 19 \text{ bit} \cdot 2048 = 38912 \text{ bit} = 4864\text{ byte} \]

        \item[4-Way Set Associative Cache]\hfill
          \[ 17 \text{ bit Tag} + 1 \text{ Valid} + 1 \text{ Dirty} + 3 \text{ bit} = 22 \text{ bit} \cdot 2048 = 45056 \text{ bit} = 5632\text{ byte} \]

        \item[Fully Associative Cache]
          \[ 28 \text{ bit Tag} + 1 \text{ Valid} + 1 \text{ Dirty} + 10\text{ bit} = 40 \text{ bit} = 5\text{ byte} \]
      \end{description}
  \end{enumerate}
\end{solution}


% 2. ERLEDIGT
\begin{problem}
  
\end{problem}
\begin{solution} \hfill
  \begin{center}
    \begin{tabular}{|c|c|c|c|c|c||c|c|c||c|c|c|}
      \hline
      t     & Adresse & Tag    & Offset & h/m & r/w & Tag    & v & a & Tag    & v & a \\ \hline \hline
      $t_0$ & -       & -      & -      & -   & -   & 000000 & 0 & 0 & 000000 & 0 & 0 \\ \hline
      $t_1$ & -       & -      & -      & -   & -   & 000000 & 0 & 0 & 000000 & 0 & 0 \\ \hline
      $t_2$ & -       & -      & -      & -   & -   & 000000 & 0 & 0 & 000000 & 0 & 0 \\ \hline
      $t_3$ & 0x23    & 001000 & 11     & m   & w   & 001000 & 0 & 0 & 000000 & 0 & 0 \\ \hline
      $t_4$ & 0x11    & 000100 & 01     & m   & w   & 001000 & 1 & 1 & 000100 & 0 & 0 \\ \hline
      $t_5$ & 0x11    & 000100 & 01     & h   & r   & 001000 & 1 & 0 & 000100 & 1 & 1 \\ \hline
      $t_6$ & 0x33    & 001100 & 11     & m   & w   & 001100 & 1 & 0 & 000100 & 1 & 1 \\ \hline
      $t_7$ & 0x11    & 000100 & 01     & h   & r   & 001100 & 1 & 1 & 000100 & 1 & 0 \\ \hline
      $t_8$ & 0x23    & 001000 & 11     & m   & r   & 001000 & 1 & 0 & 000100 & 1 & 1 \\ \hline
      $t_9$ & 0x11    & 000100 & 01     & h   & w   & 001000 & 1 & 1 & 000100 & 1 & 0 \\ \hline
    \end{tabular}
  \end{center}

  Die \emph{miss rate} ist gleich $ 4/7 \approx 57\%$
\end{solution}


% 3. ERLEDIGT
\begin{problem}
  
\end{problem}
\begin{solution}\hfill
  \begin{enumerate}[label=(\alph*)]\itemsep0pt
    \item Es können maximal 2 Blöcke $\cdot$ 4 Sets $\cdot$ 16 Datenwörter $=$ 128 byte
          an Nutzdaten gespeichert werden.

    \item \hfill\begin{center}
            \begin{tabular}{c|c|c|c|c|c|c|c|c|c}
                  & 7   & 6   & 5     & 4     & 3      & 2      & 1      & 0      &     \\ \hline
              MSB & Tag & Tag & Index & Index & Offset & Offset & Offset & Offset & LSB \\ \hline
            \end{tabular}
          \end{center}

    \item \hfill\begin{center}
            \begin{tabular}{|c|c|c|c|c|c|c|}
              \hline
              Adresse   & Adresse      &          &     &       &                  & Anzahl   \\
              (dezimal) & (binär)      & Hit/Miss & Set & Block & Inhalt           & Zugriffe \\ \hline\hline
              65        & $0100\,0001$ & Miss     & 0   & 0     & mem{[}64-79{]}   & 1        \\
              77        & $0100\,1101$ & Hit      & 0   & 0     & mem{[}64-79{]}   & 2        \\
              111       & $0110\,1111$ & Miss     & 2   & 0     & mem{[}96-111{]}  & 1        \\
              222       & $1101\,1110$ & Miss     & 1   & 0     & mem{[}208-223{]} & 1        \\
              42        & $0010\,1010$ & Miss     & 2   & 1     & mem{[}32-47{]}   & 1        \\ \hline
              121       & $0111\,1001$ & Miss     & 3   & 0     & mem{[}112-127{]} & 1        \\
              110       & $0110\,1110$ & Hit      & 2   & 0     & mem{[}96-111{]}  & 2        \\
              48        & $0011\,0000$ & Miss     & 3   & 1     & mem{[}48-63{]}   & 1        \\
              163       & $1010\,0011$ & Miss     & 2   & 1     & mem{[}160-175{]} & 1        \\
              208       & $1101\,0000$ & Hit      & 1   & 0     & mem{[}208-223{]} & 2        \\ \hline
              242       & $1111\,0010$ & Miss     & 3   & 0     & mem{[}240-255{]} & 1        \\
              220       & $1101\,1100$ & Hit      & 1   & 0     & mem{[}208-223{]} & 3        \\
              78        & $0100\,1110$ & Hit      & 0   & 0     & mem{[}64-79{]}   & 3        \\
              120       & $0111\,1000$ & Miss     & 3   & 0     & mem{[}112-127{]} & 1        \\
              51        & $0011\,0011$ & Hit      & 3   & 1     & mem{[}48-63{]}   & 2        \\ \hline
            \end{tabular}
    \end{center}

    \item Die \emph{Hit-Rate} ist $6/15 = 40\%$

    \item Es wurden $25\%$ nicht beschrieben.
  \end{enumerate}
\end{solution}


\problemnumber{5}
% 5. ERLEDIGT
\begin{problem}
  
\end{problem}
\begin{solution}\hfill
  \begin{enumerate}[label=(\alph*)]\itemsep0pt
    \item \hfill \begin{center}
                    \def\svgwidth{0.95\textwidth}\input{5a.pdf_tex}
                 \end{center}

    \item \hfill
      \[ \text{L1-Cache: } 100000\cdot 0.1 = 10000 \text{ Misses} \]
      \[ \text{L2-Cache: } 100000\cdot 0.04 = 4000 \text{ Misses} \]

    \item \hfill
    \[ 3\cdot 10^9/s = 3\cdot 10^6/ms = 3\cdot 10^3/\mu s = 3/ns \]
    \begin{align*}
      t_{L1} &= 3/3 = 1ns \\
      t_{L2} &= 45/3 = 15ns \\
      t_{main} &= 150/3 = 50ns \\
    \end{align*}

    \item \hfill
      \[ t_{eff\_L1} = 0.9\cdot 3 + 0.1\cdot 150 = 17.7 \text{ Taktzyklen} \]
      \[ t_{eff\_L2} = \underbrace{0.9\cdot 3}_\text{90\% L1} + \underbrace{0.1\cdot 0.96\cdot 45}_\text{10\% L2} + \underbrace{(1-0.9-0.1\cdot 0.96)\cdot 150}_\text{0.04\% Hauptspeicher} = 7.62 \text{ Taktzyklen} \]
      \[ 17.7\div 7.62 \approx 2.32 \]
      Die effektive Speicherzugriffszeit verbessert sich durch den L2-Cache ca.
      um das 2.32-fache.
  \end{enumerate}
    
\end{solution}


% 6. ERLEDIGT
\begin{problem}
  
\end{problem}
\begin{solution}\hfill
  \begin{center}
      \begin{tabular}{|c|c|c|c|}
        \hline
        Page-Nr & Frame-Nr & Present-Bit & Zeitpunkt    \\ \hline \hline
        000     & 00       & 0           &              \\ \hline
        001     & 10       & 1           & $t_3$        \\ \hline
        010     & 00       & 1           & $t_1$, $t_5$ \\ \hline
        011     & 00       & 0           &              \\ \hline
        100     & 11       & 1           & $t_4$        \\ \hline
        101     & 00       & 0           &              \\ \hline
        110     & 00       & 0           &              \\ \hline
        111     & 01       & 1           & $t_2$        \\ \hline
      \end{tabular}
  \end{center}

  \begin{enumerate}[label=(\alph*)]\itemsep0pt
    \item Virtuelle Adressen sind 20 Bit, physische Adressen 17 bit lang.
    \item Es sind $2^{17}$ byte physischer Speicher und $2^{20}$ byte 
          virtueller Speicher adressierbar.
    \item \hfill 
      \begin{center}
        \begin{tabular}{c|c}
          Page-Nr & Offset          \\ \hline
          00100   & 110101000111011
        \end{tabular}
      \end{center}
      Page-Nummer: 0x04\\
      Offset: 0x6A3B\\
      Frame-Nummer: 0x0\\
      physische Adresse: 0x06A3B
    \newpage
    \item \hfill
      \begin{center}
        \begin{tabular}{|c|c|c|c|}
          \hline
          Page-Nr & Frame-Nr & Present-Bit      & Zeitpunkt    \\ \hline \hline
          000     & 00       & $0\rightarrow 1$ & $t_7$        \\ \hline
          001     & 10       & 1                & $t_3$        \\ \hline
          010     & 00       & $1\rightarrow 0$ & $t_1$, $t_5$ \\ \hline
          011     & 00       & 0                &              \\ \hline
          100     & 11       & 1                & $t_4$        \\ \hline
          101     & 00       & 0                &              \\ \hline
          110     & 00       & $0\rightarrow 1$ & $t_6$        \\ \hline
          111     & 01       & $1\rightarrow 0$ & $t_2$        \\ \hline
        \end{tabular}
      \end{center}
  \end{enumerate}
\end{solution}


% 7. ERLEDIGT
\begin{problem}
  
\end{problem}
\begin{solution}\hfill
  \begin{enumerate}[label=(\alph*)]\itemsep0pt
      \item Virtuelle Adresse: 3 bit Page-Nummer, 12 bit Offset.\\
            Physische Adresse: 2 bit Frame-Nummer, 12 bit Offset.\\
            \begin{center}
              \begin{tabular}{|c|c|c|c|}
                \hline
                Page-Nr & Frame-Nr & Present-Bit & \#Zugriffe \\ \hline \hline
                000     & XX       & 0           & 000        \\ \hline
                001     & XX       & 0           & 000        \\ \hline
                010     & XX       & 0           & 000        \\ \hline
                011     & XX       & 0           & 000        \\ \hline
                100     & XX       & 0           & 000        \\ \hline
                101     & XX       & 0           & 000        \\ \hline
                110     & XX       & 0           & 000        \\ \hline
                111     & XX       & 0           & 000        \\ \hline
              \end{tabular}
            \end{center}

      \item \hfill
        \begin{center}
          \begin{tabular}{|c|c|c|}
            \hline
            Frame & Frame-Nr & phys. Adressbereich \\ \hline \hline
            0     & 00       & 0000...0FFF         \\ \hline
            1     & 01       & 1000...1FFF         \\ \hline
            2     & 10       & 2000...2FFF         \\ \hline
            3     & 11       & 3000...3FFF         \\ \hline
          \end{tabular}
        \end{center}

      \item \hfill
        \begin{center}
          \begin{tabular}{|c|c|c|}
            \hline
            \begin{tabular}[c]{@{}c@{}}Adresse\\ (hex)\end{tabular} & \begin{tabular}[c]{@{}c@{}}Adresse\\ (binär)\end{tabular} & Page-Nr \\ \hline \hline
            0x4CAD                                                  & 010|0...                                                  & 2       \\ \hline
            0x178A                                                  & 000|1...                                                  & 0       \\ \hline
            0x2431                                                  & 001|0...                                                  & 1       \\ \hline
            0x2B0B                                                  & 001|0...                                                  & 1       \\ \hline
            0x4000                                                  & 010|0...                                                  & 2       \\ \hline
            0x7DEA                                                  & 011|1...                                                  & 3       \\ \hline
            0x6BAC                                                  & 011|0...                                                  & 3       \\ \hline
            0x4FB1                                                  & 010|0...                                                  & 2       \\ \hline
          \end{tabular}
        \end{center}
        Reihenfolge der Zugriffe (Page-Nummern): $2\rightarrow 0\rightarrow 1\rightarrow 1\rightarrow 2\rightarrow 3\rightarrow 3\rightarrow 2$
        \begin{itemize}
          \item Bei den ersten drei Zugriffen treten \emph{Page-Faults} auf, die
                Pages werden in die Frames 1, 2 und 3 geladen (Frame 0 durch OS blockiert).
          \item Page 1 ist bereits geladen, erhöhe Zugriffe auf 2.
          \item Page 2 ist bereits geladen, erhöhe Zugriffe auf 2.
          \item Page 3 $\rightarrow$ \emph{Page-Fault} $\rightarrow$ kommt in Frame 2,
                weil Page 0 niedrigste Zugriffe.
          \item Page 3 bereits geladen $\rightarrow$ Zugriffe auf 2.
          \item Page 2 bereits geladen $\rightarrow$ Zugriffe auf 3.
        \end{itemize}

        Die Page-Table schaut daher zu Ende so aus:
        \begin{center}
          \begin{tabular}{|c|c|c|c|}
            \hline
            Page-Nr & Frame-Nr & Present-Bit      & \#Zugriffe                        \\ \hline \hline
            000     & 10       & $1\rightarrow 0$ & 001                               \\ \hline
            001     & 11       & 1                & $001\rightarrow 010$              \\ \hline
            010     & 01       & 1                & $001\rightarrow 010\rightarrow 011$ \\ \hline
            011     & 10       & $0\rightarrow 1$ & $001\rightarrow 010$              \\ \hline
            100     & XX       & 0                & 000                               \\ \hline
            101     & XX       & 0                & 000                               \\ \hline
            110     & XX       & 0                & 000                               \\ \hline
            111     & XX       & 0                & 000                               \\ \hline
          \end{tabular}
        \end{center}
  \end{enumerate}  
\end{solution}


\end{document}