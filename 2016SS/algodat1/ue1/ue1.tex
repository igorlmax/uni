\documentclass[a4paper, 12pt, margins=3cm]{homework}
\usepackage{tikz}

\usepackage{graphicx}
\usepackage{dsfont}
\usepackage{microtype}
\usepackage{mathrsfs}
\usepackage[ngerman]{babel}
\usepackage{csquotes}
\usepackage[T1]{fontenc}
\usepackage{lmodern}
\usepackage{wasysym}

\setlength{\parindent}{0pt}

\newcommand{\R}{\mathbb{R}}
\newcommand{\N}{\mathbb{N}}
\newcommand{\Z}{\mathbb{Z}}
\newcommand{\Q}{\mathbb{Q}}
\newcommand{\C}{\mathbb{C}}

\name{Tobias Eidelpes\\Matrikelnr.: 1527193}
\course{Algorithmen und Datenstrukturen 1\\}
\term{2016SS}
\hwnum{1}
\hwtype{Übungsblatt}
\problemtitle{Aufgabe}
\solutiontitle{Lösung}

\begin{document}

% ERLEDIGT
  \problemnumber{1}
  \begin{problem}
    
  \end{problem}
  \begin{solution} \hfill
    \begin{enumerate}\itemsep0pt
      \item W-B
      \item W-B \\ X-C
      \item W-B \\ X-C \\ Y-A
      \item W-B \\ X-C \\ Z-A
      \item W-B \\ X-C \\ Y-D \\ Z-A
    \end{enumerate}
  \end{solution}


% ERLEDIGT
  \problemnumber{2}
  \begin{problem}
   
  \end{problem}
  \begin{solution} \hfill
    \begin{enumerate}\itemsep0pt
      \item \[ 10^{12} \cdot 3.6\cdot 10^{3} = 3.6 \cdot 10^{15} \]
            \[ 100\cdot n^2 = 3.6\cdot 10^{15} \Longleftrightarrow n = \sqrt{3.6\cdot 10^{13}} \]
            \[ n = 6000000 \]

      \item \[ 10000\cdot 2^n = 3.6\cdot 10^{15} \Longleftrightarrow n = \log_2(3.6\cdot 10^{11}) \]
            \[ n \approx 38 \]

      \item \[ 2^{2^n} = 3.6\cdot 10^{15} \]
            \[ n = \log_2(\log_2(3.6\cdot 10^{15})) \]
            \[ n \approx 5 \]
    \end{enumerate}
  \end{solution}


% ERLEDIGT
  \problemnumber{3}
  \begin{problem}
    
  \end{problem}
  \begin{solution}\hfill
    \begin{enumerate}\itemsep0pt
      \item Es gilt: \[ f(n) = O(g(n))\text{ woraus folgt, dass } g(n) = \Omega(f(n)) \]
                     \[ g(n) = O(f(n))\text{ woraus folgt, dass } f(n) = \Omega(g(n)) \]

            Weil nun $f(n) = O(g(n))$ und $f(n) = \Omega(g(n))$ gilt, folgt daraus, dass
            $f(n) = \Theta(g(n))$ gilt. \\
            Umgekehrt gilt dasselbe auch für $g(n)$.

      \item $f(n) = g(n)$ gilt nicht, weil sich die beiden Funktionen um konstante
            Faktoren unterscheiden können.
    \end{enumerate}
  \end{solution}


% ERLEDIGT
  \problemnumber{5}
  \begin{problem}
    
  \end{problem}
  \begin{solution}\hfill
    \begin{center}
      \def\svgwidth{0.5\textwidth}\input{Aufgabe5.pdf_tex}
    \end{center}
  \end{solution}

\newpage

% ERLEDIGT
  \problemnumber{7}
  \begin{problem}

  \end{problem}
  \begin{solution}\hfill

        Laufzeit: $O(n)$.
        \[ r(n) = \frac{1}{3}(n^3-n) \]

        \[ A(2/2),\quad B(3/8),\quad C(4/20),\quad (5/40) \]
        Durch herumprobieren sieht man, dass das Wachstum größer als quadratisch
        sein muss, deshalb sind vier Punkte vonnöten, um eine Gleichung dritten
        Grades der Form $f(x)=ax^3+bx^2+cx+d$ aufstellen zu können. Daraus bilden
        wir vier Gleichungen:
        \[ I: 8a+4b+2c+d=2 \]
        \[ II: 27a+9b+3c+d=8 \]
        \[ III: 64a+16b+4c+d=20 \]
        \[ IV: 125a+25b+5c+d=40 \]
        Nach Anwenden des Gaußschen Eliminationsverfahrens erhalten wir für $a=\frac{1}{3}$,
        für $b=0$, für $c=-\frac{1}{3}$ und für $d=0$. Daraus folgt:
        \[ r(x) = \frac{1}{3}x^3 + 0x^2 - \frac{1}{3}x + 0 \]
        \[ r(x) = \frac{1}{3}(x^3-x) \]

  \end{solution}

\end{document}