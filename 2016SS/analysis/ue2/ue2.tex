\documentclass[a4paper, 12pt, margins=3cm]{homework}
\usepackage{tikz}

\usepackage{graphicx}
\usepackage{dsfont}
\usepackage{microtype}
\usepackage{mathrsfs}
\usepackage[ngerman]{babel}
\usepackage{csquotes}
\usepackage[T1]{fontenc}
\usepackage{lmodern}
\usepackage{wasysym}

\setlength{\parindent}{0pt}

\newcommand{\R}{\mathbb{R}}
\newcommand{\N}{\mathbb{N}}
\newcommand{\Z}{\mathbb{Z}}
\newcommand{\Q}{\mathbb{Q}}
\newcommand{\C}{\mathbb{C}}

\name{Tobias Eidelpes}
\course{Algebra und Diskrete Mathematik}
\term{2015WS}
\hwnum{5}
\hwtype{Übungsblatt}
\problemtitle{Aufgabe}
\solutiontitle{Lösung}

\begin{document}
  \begin{center}
    \textsc{Beispiele 19, 22, 34, 37, 50, 52}
  \end{center}

  \problemnumber{19}
  \begin{problem}
    Seien $(a_n)_{n\in N}$ und $(b_n)_{n\in \N}$ konvergente Folgen. Zeigen Sie,
    dass aus $a_n<b_n$ für alle $n\in \N$ immer $\lim_{n\rightarrow\infty}a_n \leq \lim_{n\rightarrow\infty}b_n$
    folgt. Lässt sich hier $\leq$ durch < ersetzen?
  \end{problem}
  \begin{solution}
    
  \end{solution}

  \problemnumber{22}
  \begin{problem}
    Man untersuche die Folge $a_n$ (mit Hilfe vollständiger Induktion) auf Monotonie
    und Beschränktheit und bestimme gegebenenfalls mit Hilfe der bekannten Rechenregeln
    für Grenzwerte den Grenzwert $\lim_{n\rightarrow\infty}a_n$. Überlegen Sie sich
    auch, warum die Folge wohldefiniert für alle $n \geq 0$ ist.
    \[ a_0 = 4, a_{n+1} = \sqrt{6a_n-9} \text{ für alle } n\geq 0 \]
  \end{problem}
  \begin{solution}
    
  \end{solution}

  \problemnumber{34}
  \begin{problem}
    Man untersuche die Folge $(a_n)_{n\in \N}$ auf Wohldefiniertheit und Konvergenz
    und bestimme gegebenenfalls den Grenzwert. (Die $a_n$ sind für fast alle $n\in \N$ definiert.)
    \[ a_n = \frac{2n^3-5n^2+7}{2n^3-5n+7} \]
  \end{problem}
  \begin{solution}
    
  \end{solution}

  \problemnumber{37}
  \begin{problem}
    Man untersuche die Folge $(a_n)_{n\in \N}$ auf Wohldefiniertheit und Konvergenz
    und bestimme gegebenenfalls den Grenzwert. (Die $a_n$ sind für fast alle $n\in \N$ definiert.)
    \[ a_n = \sqrt{n+1} - \sqrt{n} \]
  \end{problem}
  \begin{solution}
    
  \end{solution}

  \problemnumber{50}
  \begin{problem}
    Man untersuche die Folge $(a_n)_{n\geq 1}$ auf Konvergenz und bestimme gegebenenfalls
    den Grenzwert, indem man zwei geeignete Folgen $(b_n)_{n\geq 1}$, $(c_n)_{n\geq 1}$
    mit $b_n\leq a_n\leq c_n$ finde.
    \[ a_n = \frac{n^2+1}{n^3+1}+\frac{n^2+2}{n^3+2}+...+\frac{n^2+n}{n^3+n} \]
  \end{problem}
  \begin{solution}
    
  \end{solution}

  \problemnumber{52}
  \begin{problem}
    Sei die Folge $(a_n)_{n\in\N}$ rekursiv gegeben durch $a_0=0$ und
    \[ a_n = a_{n-1} + \frac{1}{n(n+1)}\quad (n\geq 1). \]
    Man zeige (mit Hilfe vollständiger Induktion) $a_n = 1-\frac{1}{n+1}$ und bestimme
    den Grenzwert.
  \end{problem}
  \begin{solution}
    
  \end{solution}
\end{document}