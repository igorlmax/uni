\documentclass[a4paper, 12pt, margins=3cm]{homework}
\usepackage{tikz}

\usepackage{graphicx}
\usepackage{dsfont}
\usepackage{microtype}
\usepackage{mathrsfs}
\usepackage[ngerman]{babel}
\usepackage{csquotes}
\usepackage[T1]{fontenc}
\usepackage{lmodern}
\usepackage{wasysym}
\usepackage{tkz-graph}

\setlength{\parindent}{0pt}

\newcommand{\R}{\mathbb{R}}
\newcommand{\N}{\mathbb{N}}
\newcommand{\Z}{\mathbb{Z}}
\newcommand{\Q}{\mathbb{Q}}
\newcommand{\C}{\mathbb{C}}

\name{Tobias Eidelpes}
\course{\\Algorithmen und Datenstrukturen 1\\}
\term{2016SS\\ Matrikelnr.: 1527193}
\hwnum{3}
\hwtype{Übungsblatt}
\problemtitle{Aufgabe}
\solutiontitle{Lösung}

\begin{document}
  \SetVertexNormal[Shape      = circle,
                 FillColor  = white,
                 LineWidth  = 1pt]

  \SetUpEdge[lw         = 1pt,
           color      = black,
           labelcolor = white,
           labeltext  = black]

  \GraphInit[vstyle=Normal] 
  \SetGraphUnit{3}


% ERLEDIGT 18
  \problemnumber{18}
  \begin{problem}
    
  \end{problem}
  \begin{solution}

    Den Graphen aus Beispiel 19 nehmend:
    \begin{center}
      \def\svgwidth{0.8\textwidth}\input{Aufgabe19.pdf_tex}
    \end{center}
    Der kürzeste Weg im ursprünglichen Graphen zwischen Knoten 2 und 4 wäre mit 10
    gewichtet. Im minimalen Spannbaum müsste ein Umweg (2-1-3-5-7-6-4) gemacht
    werden, welcher mit 26 gewichtet ist.
  \end{solution}


% ERLEDIGT 19
  \problemnumber{19}
  \begin{problem}
    
  \end{problem}
  \begin{solution}\hfill
    \begin{center}
      \def\svgwidth{0.8\textwidth}\input{Aufgabe19.pdf_tex}
    \end{center}

    Die Kanten werden nach aufsteigender Kantengewichtung ausgewählt.
    
  \end{solution}

\newpage

% ERLEDIGT 20
  \problemnumber{20}
  \begin{problem}
    
  \end{problem}
  \begin{solution}\hfill
    \begin{center}
      \def\svgwidth{0.8\textwidth}\input{Aufgabe20.pdf_tex}
    \end{center}
    \begin{center}
      \begin{tabular}{|l|l|l|c|}
        \hline
        Ausgewählt & Priority Queue      & Knotenmenge S          & Gewicht \\ \hline \hline
        2          & 1, 4, 3, 5, 6, 7, 8 & 2                      & 0       \\ \hline
        1          & 3, 4, 5, 6, 7, 8    & 2, 1                   & 8       \\ \hline
        3          & 5, 4, 6, 7, 8       & 2, 1, 3                & 13      \\ \hline
        5          & 7, 6, 4, 8          & 2, 1, 3, 5             & 19      \\ \hline
        7          & 6, 8, 4             & 2, 1, 3, 5, 7          & 20      \\ \hline
        6          & 8, 4                & 2, 1, 3, 5, 7, 6       & 24      \\ \hline
        8          & 4                   & 2, 1, 3, 5, 7, 6, 8    & 26      \\ \hline
        4          &                     & 2, 1, 3, 5, 7, 6, 8, 4 & 29      \\ \hline
      \end{tabular}
    \end{center}

    Die Kanten wurden in der Reihenfolge $8\rightarrow 5\rightarrow 6\rightarrow 1\rightarrow 4\rightarrow 2\rightarrow 3$ ausgewählt.
  \end{solution}


% ERLEDIGT 21
  \problemnumber{21}
  \begin{problem}
    
  \end{problem}
  \begin{solution}\hfill
    \begin{enumerate}[label=\alph*)]\itemsep0pt
      \item Für das Finden eines maximalen Elements ist der Max-Heap besser geeignet,
            da sich das maximale Element an der ersten Stelle befindet. In einem
            sortierten Array (welcher sich als Min-Heap qualifiziert) befindet sich
            das maximale Element am Ende.

      \item Das Löschen eines gegebenen Elements funktioniert im Max-Heap besser,
            weil in einem sortierten Array alle Elemente um eins verschoben werden
            müssen. 

      \item Das sortieren eines Arrays gelingt nicht besser als $O(n\text{ log }n)$.
            Auch das Aufbauen eines Max-Heaps ist nicht schneller als $O(n\text{ log }n)$.

      \item Das Finden eines minimalen Elements geht schneller im sortierten Array,
            weil da das kleinste Element am Anfang steht. In einem Max-Heap befindet
            sich das kleinste Element in einem der letzten Kindknoten, wodurch 
            zumindest diese durchsucht werden müssen.
    \end{enumerate}
  \end{solution}


% ERLEDIGT 22
  \problemnumber{22}
  \begin{problem}
    
  \end{problem}
  \begin{solution}\hfill
    \begin{enumerate}[label=\alph*)]\itemsep0pt
      \item Der neue Knoten wird an Position $n+1$ eingefügt:
        \begin{center}
            \begin{tikzpicture}[level/.style={sibling distance=60mm/#1}]
              \node[circle,draw](z){$10$}
                child{ node[circle,draw]{5} child{node[circle,draw] {4}} child{node[circle,draw] {2}}}
                child{ node[circle,draw]{7} child{node[circle,draw] {6}} child{node[circle,draw] {8}}};
            \end{tikzpicture}
        \end{center}
        Dann wird der neue Knoten (8) mit seinem Parent-Knoten (7) vertauscht, da er
        größer als dieser ist:
        \begin{center}
            \begin{tikzpicture}[level/.style={sibling distance=60mm/#1}]
              \node[circle,draw](z){$10$}
                child{ node[circle,draw]{5} child{node[circle,draw] {4}} child{node[circle,draw] {2}}}
                child{ node[circle,draw]{8} child{node[circle,draw] {6}} child{node[circle,draw] {7}}};
            \end{tikzpicture}
        \end{center}

      \item Der größte Knoten (10) wird gelöscht und der Knoten an der Position 
            $n$ (7) wird stattdessen eingefügt.
        \begin{center}
            \begin{tikzpicture}[level/.style={sibling distance=60mm/#1}]
              \node[circle,draw](z){$7$}
                child{ node[circle,draw]{5} child{node[circle,draw] {4}} child{node[circle,draw] {2}}}
                child{ node[circle,draw]{8} child{node[circle,draw] {6}} child[missing]{}};
            \end{tikzpicture}
        \end{center}
        Danach muss der Knoten 7 mit seinem rechten Kindknoten 8 vertauscht werden:
        \begin{center}
            \begin{tikzpicture}[level/.style={sibling distance=60mm/#1}]
              \node[circle,draw](z){$8$}
                child{ node[circle,draw]{5} child{node[circle,draw] {4}} child{node[circle,draw] {2}}}
                child{ node[circle,draw]{7} child{node[circle,draw] {6}} child[missing]{}};
            \end{tikzpicture}
        \end{center}
        Um den Knoten 2 zu löschen muss an dessen Stelle der $n$-te Knoten (6)
        geschoben werden:
        \begin{center}
            \begin{tikzpicture}[level/.style={sibling distance=60mm/#1}]
              \node[circle,draw](z){$8$}
                child{ node[circle,draw]{5} child{node[circle,draw] {4}} child{node[circle,draw] {6}}}
                child{ node[circle,draw]{7} child[missing]{} child[missing]{}};
            \end{tikzpicture}
        \end{center}
        Jetzt den Knoten 6 mit seinem Parent-Knoten vertauschen:
        \begin{center}
            \begin{tikzpicture}[level/.style={sibling distance=60mm/#1}]
              \node[circle,draw](z){$8$}
                child{ node[circle,draw]{6} child{node[circle,draw] {4}} child{node[circle,draw] {5}}}
                child{ node[circle,draw]{7} child[missing]{} child[missing]{}};
            \end{tikzpicture}
        \end{center}
    \end{enumerate}
    
  \end{solution}


% ERLEDIGT 23
  \problemnumber{23}
  \begin{problem}

  \end{problem}
  \begin{solution}\hfill
    \begin{enumerate}[label=\alph*)]\itemsep0pt
      \item \hfill
        \begin{center}
          \begin{tikzpicture}
            \Vertex{A}
            \NOEA(A){C}
            \SOEA(C){B}
            \Edge[label=$2$](A)(C)
            \Edge[label=$-1$](C)(B)
            \Edge[label=$4$](A)(B)
          \end{tikzpicture}
        \end{center}
        Ausgehend vom Startknoten A findet der Dijkstra-Algorithmus den richtigen
        Weg von A nach C. 
        \newpage
      \item \hfill
        \begin{center}
          \begin{tikzpicture}
            \Vertex{A}
            \NOEA(A){C}
            \SOEA(C){B}
            \Edge[label=$1$](A)(C)
            \Edge[label=$-2$](C)(B)
            \Edge[label=$2$](A)(B)
          \end{tikzpicture}
        \end{center}
        Wiederum ausgehend von A findet der Algorithmus den falschen Weg von A
        nach C, weil hier der Weg über B kürzer wäre. A-C hat das Gewicht 2 und 
        A-B-C hat das Gewicht 0.
    \end{enumerate}
  \end{solution}
  

\end{document}