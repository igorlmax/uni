\documentclass[a4paper, 12pt, margins=3cm]{homework}
\usepackage{tikz}

\usepackage{graphicx}
\usepackage{dsfont}
\usepackage{microtype}
\usepackage{mathrsfs}
\usepackage[ngerman]{babel}
\usepackage{csquotes}
\usepackage[T1]{fontenc}
\usepackage{lmodern}
\usepackage{wasysym}



\setlength{\parindent}{0pt}

\newcommand{\R}{\mathbb{R}}
\newcommand{\N}{\mathbb{N}}
\newcommand{\Z}{\mathbb{Z}}
\newcommand{\Q}{\mathbb{Q}}
\newcommand{\C}{\mathbb{C}}

\name{Tobias Eidelpes}
\course{\\Algorithmen und Datenstrukturen 1\\}
\term{2016SS \\ Matrikelnr.: 1527193}
\hwnum{2}
\hwtype{Übungsblatt}
\problemtitle{Aufgabe}
\solutiontitle{Lösung}

\begin{document}


% ERLEDIGT
  \problemnumber{9}
  \begin{problem}

  \end{problem}
  \begin{solution}\hfill
    \subsubsection*{Selection-Sort}
      \[ [|17, 24, 13, 5, 9] \]
      \[ [5\,|\, 24, 13, 17, 9] \]
      \[ [5, 9\,|\, 13, 17, 24] \]
      \[ [5, 9, 13\,|\, 17, 24] \]
      \[ [5, 9, 13, 17\,|\, 24] \]
      Der Algorithmus führt in diesem Fall zehn Vergleichsoperationen durch.

    \subsubsection*{Insertion-Sort}
      \[ [17, 24, 13, 5, 9] \]
      \[ [13, 17, 24, 5, 9] \]
      \[ [5, 13, 17, 24, 9] \]
      \[ [5, 9, 13, 17, 24] \]
      Der Algorithmus führt in diesem Fall acht Vergleichsoperationen durch.

  \end{solution}


% ERLEDIGT
  \problemnumber{10}
  \begin{problem}
    
  \end{problem}
  \begin{solution}\hfill

    \subsubsection*{Breitensuche}
      \begin{enumerate}
        \item \[ A\rightarrow B\rightarrow D\rightarrow I \]
        \item \[ \rightarrow C\rightarrow E\rightarrow G\rightarrow J \]
        \item \[ \rightarrow F\rightarrow H \]
      \end{enumerate}

    \subsubsection*{Tiefensuche}
        \[ A\rightarrow B\rightarrow C\rightarrow E\rightarrow D\rightarrow G\rightarrow H\rightarrow
                 F\rightarrow I\rightarrow J \]

  \end{solution}


% ERLEDIGT
  \problemnumber{11}
  \begin{problem}
    
  \end{problem}
  \begin{solution}
    \[ H\rightarrow A\rightarrow B\rightarrow D\rightarrow E\rightarrow G
       \rightarrow I\rightarrow J\rightarrow C\rightarrow F \]
  \end{solution}


% ERLEDIGT
  \problemnumber{13}
  \begin{problem}
    
  \end{problem}
  \begin{solution}\hfill 
    \subsection*{Adjazenzmatrix}
    \[ 
      \bordermatrix {
        ~ & A & B & C & D & E & F & G & H & I & J \cr
        A & 0 & 1 & 0 & 1 & 0 & 0 & 0 & 0 & 1 & 0 \cr
        B & 1 & 0 & 1 & 1 & 1 & 0 & 0 & 0 & 0 & 0 \cr
        C & 0 & 1 & 0 & 0 & 1 & 1 & 0 & 0 & 0 & 0 \cr
        D & 1 & 1 & 0 & 0 & 1 & 0 & 1 & 0 & 0 & 0 \cr
        E & 0 & 1 & 1 & 1 & 0 & 1 & 1 & 1 & 0 & 0 \cr
        F & 0 & 0 & 1 & 0 & 1 & 0 & 0 & 1 & 0 & 0 \cr
        G & 0 & 0 & 0 & 1 & 1 & 0 & 0 & 1 & 1 & 1 \cr
        H & 0 & 0 & 0 & 0 & 1 & 1 & 1 & 0 & 0 & 1 \cr
        I & 1 & 0 & 0 & 0 & 0 & 0 & 1 & 0 & 0 & 1 \cr
        J & 0 & 0 & 0 & 0 & 0 & 0 & 1 & 1 & 1 & 0
      } 
    \]

    \subsection*{Adjazenzlisten}
    \begin{align*}
      A & : B,\; D,\; I \\
      B & : A,\; C,\; D,\; E \\
      C & : B,\; D,\; E,\; F \\
      D & : A,\; B,\; E,\; G \\
      E & : B,\; C,\; D,\; F,\; G,\; H \\
      F & : C,\; E,\; G,\; H \\
      G & : D,\; E,\; H,\; I,\; J \\
      H & : E,\; F,\; G,\; J \\
      I & : A,\; G,\; J \\
      J & : G,\; H,\; I
    \end{align*}
  \end{solution}


\end{document}