\documentclass[a4paper, 12pt, margins=3cm]{homework}
\usepackage{tikz}

\usepackage{graphicx}
\usepackage{dsfont}
\usepackage{microtype}
\usepackage{mathrsfs}
\usepackage[ngerman]{babel}
\usepackage{csquotes}
\usepackage[T1]{fontenc}
\usepackage{lmodern}
\usepackage{wasysym}



\setlength{\parindent}{0pt}

\newcommand{\R}{\mathbb{R}}
\newcommand{\N}{\mathbb{N}}
\newcommand{\Z}{\mathbb{Z}}
\newcommand{\Q}{\mathbb{Q}}
\newcommand{\C}{\mathbb{C}}

\name{Tobias Eidelpes}
\course{\\Algorithmen und Datenstrukturen 1\\}
\term{2016SS \\ Matrikelnr.: 1527193}
\hwnum{2}
\hwtype{Übungsblatt}
\problemtitle{Aufgabe}
\solutiontitle{Lösung}

\begin{document}


% AUSSTÄNDIG
  \problemnumber{9}
  \begin{problem}
    Sortieren Sie das Array A = [17, 24, 13, 5, 9] jeweils mittels Selection-Sort
    und Insertion-Sort entsprechend den Algorithmen aus der Vorlesung. Stellen Sie 
    die einzelnen Zwischenschritte, die die Algorithmen ausführen, dar. Es genügt das Array nach
    jeder Iteration der äußersten Schleife abzubilden. Wie viele Vergleichsoperationen führt
    jeder der Algorithmen durch?
  \end{problem}
  \begin{solution}\hfill
    \subsubsection*{Selection-Sort}
      \[ [|17, 24, 13, 5, 9] \]
      \[ [5\,|\, 24, 13, 17, 9] \]
      \[ [5, 9\,|\, 13, 17, 24] \]
      \[ [5, 9, 13\,|\, 17, 24] \]
      \[ [5, 9, 13, 17\,|\, 24] \]
      Der Algorithmus führt in diesem Fall zehn Vergleichsoperationen durch.

    \subsubsection*{Insertion-Sort}
      \[ [17, 24, 13, 5, 9] \]
      \[ [13, 17, 24, 5, 9] \]
      \[ [5, 13, 17, 24, 9] \]
      \[ [5, 9, 13, 17, 24] \]


  \end{solution}

\newpage

% AUSSTÄNDIG
  \problemnumber{10}
  \begin{problem}
    
  \end{problem}
  \begin{solution}\hfill

    \subsubsection*{Breitensuche}
      \begin{enumerate}
        \item \[ A\rightarrow B\rightarrow D\rightarrow I \]
        \item \[ \rightarrow C\rightarrow E\rightarrow G\rightarrow J \]
        \item \[ \rightarrow F\rightarrow H \]
      \end{enumerate}

    \subsubsection*{Tiefensuche}
        \[ A\rightarrow B\rightarrow C\rightarrow E\rightarrow D\rightarrow G\rightarrow H\rightarrow
                 F\rightarrow I\rightarrow J \]

  \end{solution}


% AUSSTÄNDIG
  \problemnumber{11}
  \begin{problem}
    
  \end{problem}
  \begin{solution}
    
  \end{solution}


% AUSSTÄNDIG
  \problemnumber{12}
  \begin{problem}
    
  \end{problem}
  \begin{solution}
    
  \end{solution}


% AUSSTÄNDIG
  \problemnumber{13}
  \begin{problem}
    
  \end{problem}
  \begin{solution}\hfill 
    \subsection*{Adjazenzmatrix}
    \[ 
      \bordermatrix {
        ~ & A & B & C & D & E & F & G & H & I & J \cr
        A & 0 & 1 & 0 & 1 & 0 & 0 & 0 & 0 & 1 & 0 \cr
        B & 1 & 0 & 1 & 1 & 1 & 0 & 0 & 0 & 0 & 0 \cr
        C & 0 & 1 & 0 & 0 & 1 & 1 & 0 & 0 & 0 & 0 \cr
        D & 1 & 1 & 0 & 0 & 1 & 0 & 1 & 0 & 0 & 0 \cr
        E & 0 & 1 & 1 & 1 & 0 & 1 & 1 & 1 & 0 & 0 \cr
        F & 0 & 0 & 1 & 0 & 1 & 0 & 0 & 1 & 0 & 0 \cr
        G & 0 & 0 & 0 & 1 & 1 & 0 & 0 & 1 & 1 & 1 \cr
        H & 0 & 0 & 0 & 0 & 1 & 1 & 1 & 0 & 0 & 1 \cr
        I & 1 & 0 & 0 & 0 & 0 & 0 & 1 & 0 & 0 & 1 \cr
        J & 0 & 0 & 0 & 0 & 0 & 0 & 1 & 1 & 1 & 0
      } 
    \]

    \subsection*{Adjazenzlisten}
    \begin{align*}
      A & : B,\; D,\; I \\
      B & : A,\; C,\; D,\; E \\
      C & : B,\; D,\; E,\; F \\
      D & : A,\; B,\; E,\; G \\
      E & : B,\; C,\; D,\; F,\; G,\; H \\
      F & : C,\; E,\; G,\; H \\
      G & : D,\; E,\; H,\; I,\; J \\
      H & : E,\; F,\; G,\; J \\
      I & : A,\; G,\; J \\
      J & : G,\; H,\; I
    \end{align*}
  \end{solution}


% AUSSTÄNDIG
  \problemnumber{14}
  \begin{problem}
    
  \end{problem}
  \begin{solution}
    
  \end{solution}


% AUSSTÄNDIG
  \problemnumber{15}
  \begin{problem}

  \end{problem}
  \begin{solution}
    
  \end{solution}


% AUSSTÄNDIG
  \problemnumber{16}
  \begin{problem}
    
  \end{problem}
  \begin{solution}
    
  \end{solution}

\end{document}