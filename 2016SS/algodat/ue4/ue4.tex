\documentclass[a4paper, 12pt, margins=2.5cm]{homework}
\usepackage{tikz}

\usepackage{graphicx}
\usepackage{dsfont}
\usepackage{microtype}
\usepackage{mathrsfs}
\usepackage[ngerman]{babel}
\usepackage{csquotes}
\usepackage[T1]{fontenc}
\usepackage{lmodern}
\usepackage{wasysym}
\usepackage{listings}

\setlength{\parindent}{0pt}

\newcommand{\R}{\mathbb{R}}
\newcommand{\N}{\mathbb{N}}
\newcommand{\Z}{\mathbb{Z}}
\newcommand{\Q}{\mathbb{Q}}
\newcommand{\C}{\mathbb{C}}

\newcommand{\onenode}[1]{%
  \begin{tabular}{|c|}
    \hline #1\\ \hline
  \end{tabular}
}
\newcommand{\twonode}[2]{%
  \begin{tabular}{|c|c|}
    \hline #1 & #2\\ \hline
  \end{tabular}
}
\newcommand{\threenode}[3]{%
  \begin{tabular}{|c|c|c|}
    \hline #1 & #2 & #3\\ \hline
  \end{tabular}
}
\newcommand{\fournode}[4]{%
  \begin{tabular}{|c|c|c|c|}
    \hline #1 & #2 & #3 & #4\\ \hline
  \end{tabular}
}
\newcommand{\fivenode}[5]{%
  \begin{tabular}{|c|c|c|c|c|}
    \hline #1 & #2 & #3 & #4 & #5\\ \hline
  \end{tabular}
}

\lstnewenvironment{algorithm}[1][] %defines the algorithm listing environment
{   
    \lstset{ %this is the stype
        frame=TB,
        numbers=left, 
        numberstyle=\scriptsize,
        basicstyle=\small, 
        keywordstyle=\color{blue}\bfseries,
        keywords={, return, elif, function, then, in, if, else, foreach, while, begin, end, for, or, } %add the keywords you want, or load a language as Rubens explains in his comment above.
        numbers=left,
        xleftmargin=.04\textwidth,
        #1 % this is to add specific settings to an usage of this environment (for instnce, the caption and referable label)
    }
}
{}

\setlength{\tabcolsep}{2pt}

\name{Tobias Eidelpes}
\course{\\Algorithmen und Datenstrukturen 1\\}
\term{2016SS\\ Matrikelnr.: 1527193}
\hwnum{4}
\hwtype{Übungsblatt}
\problemtitle{Aufgabe}
\solutiontitle{Lösung}

\begin{document}

% ERLEDIGT 27
  \problemnumber{27}
  \begin{problem}
    
  \end{problem}
  \begin{solution} \hfill
    \begin{center}
      \begin{tabular}{l||c|c|c|c|c|c|c|c|c|c|c|c|c}
                                & 0  & 1 & 2 & 3 & 4 & 5  & 6 & 7  & 8  & 9  & 10 & 11 & 12 \\ \hline \hline
        Lineares Sondieren      &    &   & 2 &   &   & 18 & 6 & 20 & 32 & 33 & 45 &    &    \\ \hline
        Quadratisches Sondieren &    &   & 2 &   &   & 18 & 6 & 20 &    & 32 & 33 &    & 45 \\ \hline
        Double Hashing          & 33 &   & 2 &   &   & 18 & 6 & 20 &    &    & 45 & 32 &   
      \end{tabular}
    \end{center}
  \end{solution}


% ERLEDIGT 30
  \problemnumber{30}
  \begin{problem}
    
  \end{problem}
  \begin{solution}\hfill
    \begin{enumerate}[label=(\alph*)]\itemsep0pt
      \item
        \begin{enumerate}[label=(\arabic*)]\itemsep0pt
          \item Nicht möglich, da 357 kleiner als 364 ist und die Suche an dieser
                Stelle nach links geht anstatt nach rechts.
          \item Nicht möglich, da 212 kleiner als 292 ist und sich im rechten Subtree
                von 292 befindet.
          \item Nicht möglich, da 898 links von 897 ist und damit die Vorraussetzung,
                dass alle Elemente im linken Subtree kleiner sind als im Wurzelknoten
                verletzt worden ist.
          \item Möglich.
        \end{enumerate} 

      \item \hfill

        \begin{algorithm}[mathescape]
function validate(list[], target, min, max):
  for i in list:
    if i < min or i > max then
      return 0
    elif i > target then
      max $\gets$ i
    elif i < target then
      min $\gets$ i
    elif i $==$ target then
      return 1
    else:
      return 0
  return 1
\end{algorithm}
    \end{enumerate}
  \end{solution}


% ERLEDIGT 31
  \problemnumber{31}
  \begin{problem}

  \end{problem}
  \begin{solution}\hfill
    \begin{enumerate}[label=(\alph*)]\itemsep0pt
      \item \hfill

        \begin{center}
          \begin{tikzpicture}[level/.style={sibling distance=40mm/#1}]
            \node[circle,draw](z){$20$};
          \end{tikzpicture}
        \end{center}
        Starte mit 20 als einzigem Knoten.

        \begin{center}
          \begin{tikzpicture}[level/.style={sibling distance=40mm/#1}]
            \node[circle,draw](z){$20$}
              child[missing]
              child{ node[circle,draw]{50}};
          \end{tikzpicture}
        \end{center}
        Füge 50 als rechten Knoten von 20 ein.
        
        \begin{center}
          \begin{tikzpicture}[level/.style={sibling distance=40mm/#1}]
            \node[circle,draw](z){$20$}
              child[missing]
              child { 
                node[circle, draw] {50}
                child[missing]
                child {
                  node[circle, draw] {70}
                }
              };
          \end{tikzpicture}
        \end{center}
        Füge 70 als rechten Knoten von 50 ein. Überprüfe die Balance $\rightarrow$
        mehr als 2 Unterschied $\rightarrow$ rebalance.

        \begin{center}
          \begin{tikzpicture}[level/.style={sibling distance=40mm/#1}]
            \node[circle,draw](z){$50$}
              child {
                node[circle, draw] {20}
              }
              child { 
                node[circle, draw] {70}
              };
          \end{tikzpicture}
        \end{center}
        Rotiere den Graphen einmal nach links, sodass 50 in der Wurzel steht.

        \begin{center}
          \begin{tikzpicture}[level/.style={sibling distance=40mm/#1}]
            \node[circle,draw](z){$50$}
              child {
                node[circle, draw] {20}
                child {
                  node[circle, draw] {10}
                }
                child[missing]
              }
              child { 
                node[circle, draw] {70}
              };
          \end{tikzpicture}
        \end{center}
        Füge 10 links von 20 ein, da kleiner als 50 und kleiner als 20. Kein balancing
        vonnöten, da Höhenunterschied nicht größer als 1.

        \begin{center}
          \begin{tikzpicture}[level/.style={sibling distance=40mm/#1}]
            \node[circle,draw](z){$50$}
              child {
                node[circle, draw] {20}
                child {
                  node[circle, draw] {10}
                }
                child {
                  node[circle, draw] {40}
                }
              }
              child { 
                node[circle, draw] {70}
              };
          \end{tikzpicture}
        \end{center}
        40 ist kleiner als 50, daher gehe in linken Subtree. Größer als 20, daher 
        in rechten Subtree von 20. Noch kein Element vorhanden und an dieser Stelle
        einfügen. Balancing wieder keines notwendig, da Höhenunterschied nicht
        größer als 1.

        \begin{center}
          \begin{tikzpicture}[level/.style={sibling distance=40mm/#1}]
            \node[circle,draw](z){$50$}
              child {
                node[circle, draw] {20}
                child {
                  node[circle, draw] {10}
                }
                child {
                  node[circle, draw] {40}
                  child {
                    node[circle, draw] {30}
                  }
                  child[missing]
                }
              }
              child { 
                node[circle, draw] {70}
              };
          \end{tikzpicture}
        \end{center}
        Füge 30 in linken Subtree von 40 ein, da kleiner als 50, größer als 20
        und kleiner als 40. Höhenunterschied von linkem und rechtem Subtree von 50
        ist größer als 1, daher balancing notwendig.

        \begin{center}
          \begin{tikzpicture}[level/.style={sibling distance=40mm/#1}]
            \node[circle,draw](z){$40$}
              child {
                node[circle, draw] {20}
                child {
                  node[circle, draw] {10}
                }
                child {
                  node[circle, draw] {30}
                }
              }
              child { 
                node[circle, draw] {50}
                child[missing]
                child {
                  node[circle, draw] {70}
                }
              };
          \end{tikzpicture}
        \end{center}
        Hier wurde 40 zweimal nach rechts rotiert.

        \begin{center}
          \begin{tikzpicture}[level/.style={sibling distance=40mm/#1}]
            \node[circle,draw](z){$40$}
              child {
                node[circle, draw] {20}
                child {
                  node[circle, draw] {10}
                }
                child {
                  node[circle, draw] {30}
                }
              }
              child { 
                node[circle, draw] {50}
                child[missing]
                child {
                  node[circle, draw] {70}
                  child {
                    node[circle, draw] {60}
                  }
                  child[missing]
                }
              };
          \end{tikzpicture}
        \end{center}
        Höhenunterschied ist zwischen den Subtrees von 50 größer als 1, daher 
        balancing notwendig.

        \begin{center}
          \begin{tikzpicture}[level/.style={sibling distance=40mm/#1}]
            \node[circle,draw](z){$40$}
              child {
                node[circle, draw] {20}
                child {
                  node[circle, draw] {10}
                }
                child {
                  node[circle, draw] {30}
                }
              }
              child { 
                node[circle, draw] {60}
                child {
                  node[circle, draw] {50}
                }
                child {
                  node[circle, draw] {70}
                }
              };
          \end{tikzpicture}
        \end{center}
        Endgültiger AVL-Baum.

      \item \hfill
        \begin{center}
          \begin{tikzpicture}[level/.style={sibling distance=80mm/#1}]
            \node[circle,draw](z){$20$}
              child {
                node[circle, draw] {17}
                child {
                  node[circle, draw] {5}
                  child {
                    node[circle, draw] {3}
                  }
                  child {
                    node[circle, draw] {7}
                    child {
                      node[circle, draw] {6}
                    }
                    child[missing]
                  }
                }
                child {
                  node[circle, draw] {19}
                  child {
                    node[circle, draw] {18}
                  }
                  child[missing]
                }
              }
              child { 
                node[circle, draw] {30}
                child {
                  node[circle, draw] {25}
                  child {
                    node[circle, draw] {24}
                  }
                  child[missing]
                }
                child {
                  node[circle, draw] {33}
                }
              };
          \end{tikzpicture}
        \end{center}
        Nun 33 löschen.

        \begin{center}
          \begin{tikzpicture}[level/.style={sibling distance=80mm/#1}]
            \node[circle,draw](z){$20$}
              child {
                node[circle, draw] {17}
                child {
                  node[circle, draw] {5}
                  child {
                    node[circle, draw] {3}
                  }
                  child {
                    node[circle, draw] {7}
                    child {
                      node[circle, draw] {6}
                    }
                    child[missing]
                  }
                }
                child {
                  node[circle, draw] {19}
                  child {
                    node[circle, draw] {18}
                  }
                  child[missing]
                }
              }
              child { 
                node[circle, draw] {30}
                child {
                  node[circle, draw] {25}
                  child {
                    node[circle, draw] {24}
                  }
                  child[missing]
                }
                child[missing]
              };
          \end{tikzpicture}
        \end{center}
        Höhe von linkem Subtree von 30 ist um zwei größer als rechter Subtree 
        von 30. $\rightarrow$ rotate right.

        \begin{center}
          \begin{tikzpicture}[level/.style={sibling distance=80mm/#1}]
            \node[circle,draw](z){$20$}
              child {
                node[circle, draw] {17}
                child {
                  node[circle, draw] {5}
                  child {
                    node[circle, draw] {3}
                  }
                  child {
                    node[circle, draw] {7}
                    child {
                      node[circle, draw] {6}
                    }
                    child[missing]
                  }
                }
                child {
                  node[circle, draw] {19}
                  child {
                    node[circle, draw] {18}
                  }
                  child[missing]
                }
              }
              child { 
                node[circle, draw] {25}
                child {
                  node[circle, draw] {24}
                }
                child {
                  node[circle, draw] {30}
                }
              };
          \end{tikzpicture}
        \end{center}
        Linker Subtree ist um mehr als 1 größer als rechter Subtree.

        \begin{center}
          \begin{tikzpicture}[level/.style={sibling distance=80mm/#1}]
            \node[circle,draw](z){$17$}
              child {
                node[circle, draw] {5}
                child {
                  node[circle, draw] {3}
                }
                child {
                  node[circle, draw] {7}
                  child {
                    node[circle, draw] {6}
                  }
                  child[missing]
                }
              }
              child { 
                node[circle, draw] {20}
                child {
                  node[circle, draw] {19}
                  child {
                    node[circle, draw] {18}
                  }
                  child[missing]
                }
                child {
                  node[circle, draw] {25}
                  child {
                    node[circle, draw] {24}
                  }
                  child {
                    node[circle, draw] {30}
                  }
                }
              };
          \end{tikzpicture}
        \end{center}
        Endgültiger AVL-Baum.

    \end{enumerate}
  \end{solution}


% ERLEDIGT 32
  \problemnumber{32}
  \begin{problem}
    
  \end{problem}
  \begin{solution}\hfill
    \begin{enumerate}[label=(\alph*)]\itemsep0pt
      \item \hfill
        \begin{center}
          \begin{tikzpicture}[
            level distance=50pt,
            every node/.style={inner sep=0pt,font=\large},
            bad/.style={fill=red!50},
            level 1/.style={sibling distance=80pt},
            level 2/.style={sibling distance=50pt}
          ]
            \node {\onenode{1}};
          \end{tikzpicture}
        \end{center}

        \begin{center}
          \begin{tikzpicture}[
            level distance=50pt,
            every node/.style={inner sep=0pt,font=\large},
            bad/.style={fill=red!50},
            level 1/.style={sibling distance=80pt},
            level 2/.style={sibling distance=50pt}
          ]
            \node {\twonode{1}{3}};
          \end{tikzpicture}
        \end{center}

        \begin{center}
          \begin{tikzpicture}[
            level distance=50pt,
            every node/.style={inner sep=0pt,font=\large},
            bad/.style={fill=red!50},
            level 1/.style={sibling distance=80pt},
            level 2/.style={sibling distance=50pt}
          ]
            \node[bad] {\threenode{1}{3}{6}};
          \end{tikzpicture}
        \end{center}

        \begin{center}
          \begin{tikzpicture}[
            level distance=50pt,
            every node/.style={inner sep=0pt,font=\large},
            bad/.style={fill=red!50},
            level 1/.style={sibling distance=80pt},
            level 2/.style={sibling distance=50pt}
          ]
            \node {\onenode{3}}
              child {node {\onenode{1}}}
              child {node {\onenode{6}}};
          \end{tikzpicture}
        \end{center}

        \begin{center}
          \begin{tikzpicture}[
            level distance=50pt,
            every node/.style={inner sep=0pt,font=\large},
            bad/.style={fill=red!50},
            level 1/.style={sibling distance=80pt},
            level 2/.style={sibling distance=50pt}
          ]
            \node {\onenode{3}}
              child {node {\onenode{1}}}
              child {node {\twonode{6}{8}}};
          \end{tikzpicture}
        \end{center}

        \begin{center}
          \begin{tikzpicture}[
            level distance=50pt,
            every node/.style={inner sep=0pt,font=\large},
            bad/.style={fill=red!50},
            level 1/.style={sibling distance=80pt},
            level 2/.style={sibling distance=50pt}
          ]
            \node {\onenode{3}}
              child {node {\onenode{1}}}
              child {node[bad] {\threenode{6}{8}{11}}};
          \end{tikzpicture}
        \end{center}

        \begin{center}
          \begin{tikzpicture}[
            level distance=50pt,
            every node/.style={inner sep=0pt,font=\large},
            bad/.style={fill=red!50},
            level 1/.style={sibling distance=80pt},
            level 2/.style={sibling distance=50pt}
          ]
            \node {\twonode{3}{8}}
              child {node {\onenode{1}}}
              child {node {\onenode{6}}}
              child {node {\onenode{11}}};
          \end{tikzpicture}
        \end{center}

        \begin{center}
          \begin{tikzpicture}[
            level distance=50pt,
            every node/.style={inner sep=0pt,font=\large},
            bad/.style={fill=red!50},
            level 1/.style={sibling distance=80pt},
            level 2/.style={sibling distance=50pt}
          ]
            \node {\twonode{3}{8}}
              child {node {\onenode{1}}}
              child {node {\onenode{6}}}
              child {node {\twonode{11}{14}}};
          \end{tikzpicture}
        \end{center}

        \begin{center}
          \begin{tikzpicture}[
            level distance=50pt,
            every node/.style={inner sep=0pt,font=\large},
            bad/.style={fill=red!50},
            level 1/.style={sibling distance=80pt},
            level 2/.style={sibling distance=50pt}
          ]
            \node {\twonode{3}{8}}
              child {node {\onenode{1}}}
              child {node {\onenode{6}}}
              child {node[bad] {\threenode{11}{14}{20}}};
          \end{tikzpicture}
        \end{center}

        \begin{center}
          \begin{tikzpicture}[
            level distance=50pt,
            every node/.style={inner sep=0pt,font=\large},
            bad/.style={fill=red!50},
            level 1/.style={sibling distance=80pt},
            level 2/.style={sibling distance=50pt}
          ]
            \node[bad] {\threenode{3}{8}{14}}
              child {node {\onenode{1}}}
              child {node {\onenode{6}}}
              child {node {\onenode{11}}}
              child {node {\onenode{20}}};
          \end{tikzpicture}
        \end{center}

        \begin{center}
          \begin{tikzpicture}[
            level distance=50pt,
            every node/.style={inner sep=0pt,font=\large},
            bad/.style={fill=red!50},
            level 1/.style={sibling distance=80pt},
            level 2/.style={sibling distance=50pt}
          ]
            \node {\onenode{8}}
              child {node {\onenode{3}}
                child {node {\onenode{1}}}
                child {node {\onenode{6}}}
              }
              child {node {\onenode{14}}
                child {node {\onenode{11}}}
                child {node {\onenode{20}}}
              }
              ;
          \end{tikzpicture}
        \end{center}

        \begin{center}
          \begin{tikzpicture}[
            level distance=50pt,
            every node/.style={inner sep=0pt,font=\large},
            bad/.style={fill=red!50},
            level 1/.style={sibling distance=80pt},
            level 2/.style={sibling distance=50pt}
          ]
            \node {\onenode{8}}
              child {node {\onenode{3}}
                child {node {\onenode{1}}}
                child {node {\onenode{6}}}
              }
              child {node {\onenode{14}}
                child {node {\onenode{11}}}
                child {node {\twonode{20}{40}}}
              }
              ;
          \end{tikzpicture}
        \end{center}

        \begin{center}
          \begin{tikzpicture}[level/.style={sibling distance=40mm/#1}]
            \node[circle,draw](z){$1$}
              child[missing]
              child{ node[circle,draw]{3}
                child[missing]
                child{ node[circle,draw]{6}
                  child[missing]
                  child{ node[circle,draw]{8}
                    child[missing]
                    child{ node[circle,draw]{11}
                      child[missing]
                      child{ node[circle,draw]{14}
                        child[missing]
                        child{ node[circle,draw]{20}
                          child[missing]
                          child{ node[circle,draw]{40}
                          }
                        }
                      }
                    }
                  }
                }
              };
          \end{tikzpicture}
        \end{center}

      \item \hfill
        \begin{center}
          \begin{tikzpicture}[
            level distance=50pt,
            every node/.style={inner sep=0pt,font=\large},
            bad/.style={fill=red!50},
            level 1/.style={sibling distance=80pt},
            level 2/.style={sibling distance=50pt}
          ]
            \node {\twonode{6}{20}}
              child {node {\twonode{1}{3}}}
              child {node {\onenode{14}}}
              child {node {\onenode{40}}};
          \end{tikzpicture}
        \end{center}
    \end{enumerate}
    
  \end{solution}

\end{document}