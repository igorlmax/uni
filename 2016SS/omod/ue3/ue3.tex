\documentclass[a4paper, 12pt, margins=3cm]{homework}
\usepackage{tikz}

\usepackage{graphicx}
\usepackage{dsfont}
\usepackage{microtype}
\usepackage{mathrsfs}
\usepackage[ngerman]{babel}
\usepackage{csquotes}
\usepackage[T1]{fontenc}
\usepackage{lmodern}
\usepackage{wasysym}

\setlength{\parindent}{0pt}

\newcommand{\R}{\mathbb{R}}
\newcommand{\N}{\mathbb{N}}
\newcommand{\Z}{\mathbb{Z}}
\newcommand{\Q}{\mathbb{Q}}
\newcommand{\C}{\mathbb{C}}

\name{Tobias Eidelpes}
\course{Objektorientierte Modellierung}
\term{2016SS}
\hwnum{3}
\hwtype{Übungsblatt}
\problemtitle{Aufgabe}
\solutiontitle{Lösung}

\begin{document}

% AUSSTÄNDIG
  \problemnumber{1}
  \begin{problem}
    
  \end{problem}
  \begin{solution} \hfill
        \begin{description}
          \item[Sequenzdiagramm]
        \end{description}
  \end{solution}


% AUSSTÄNDIG
  \problemnumber{2}
  \begin{problem}
    
  \end{problem}
  \begin{solution}
    
  \end{solution}


% AUSSTÄNDIG
  \problemnumber{3}
  \begin{problem}
    
  \end{problem}
  \begin{solution}
    
  \end{solution}


% AUSSTÄNDIG
  \problemnumber{4}
  \begin{problem}
    
  \end{problem}
  \begin{solution}
    
  \end{solution}


% AUSSTÄNDIG
  \problemnumber{5}
  \begin{problem}
    
  \end{problem}
  \begin{solution}
    
  \end{solution}


% AUSSTÄNDIG
  \problemnumber{6}
  \begin{problem}
    
  \end{problem}
  \begin{solution}
    
  \end{solution}

\end{document}